\subsection{Vertailu}
Seuraavaksi käydään läpi edellä esitettyjen reuna-arkkitehtuuriehdotuksien ominaisuuksia ja käsitellään niiden vaikutuksia itse toteutettavaan järjestelmään.
Kunkin arkkitehtuuriehdotuksen ominaisuusjoukko ohjaa toteutusta erilaisiin ratkaisuihin. Taulukossa \ref{table:features} on tiivistetysti koottu kunkin arkkitehtuuriratkaisun ominaisuudet.
Lopuksi käsitellään reuna-arkkitehtuurien yhteneväisyyksiä ja eroja ETSI:n MEC spesifikaation kanssa.

On tärkeää huomata että käsiteltyjen ominaisuuksien välillä on riippuvuksia. Tämän seurauksena jonkin ominaisuuden toteuttaminen tietyllä tavalla saattaa estää tai rajoittaa joidenkin toiminnallisuuksien toteuttamista. 


Reuna-arkkitehtuurien keskeisin ominaisuus on tapa jolla järjestelmä integroituu osaksi mobiiliverkkoa. Kaikki muut ominaisuudet vaikuttavat olevan riippuvaisia tästä.
Kappaleessa \ref{integraatio} esitellyn jaon mukaan SCC ja CONCERT edustavat suoraa integraatiota, SMORE ja MobiScud edustavat läpinäkyvää integraatiota ja FMC edustaa epäsuoraa.

SCC liittyy mobiiliverkon toimintoihin niin sanotusti natiivien yhteyksien avulla, joka edellyttää uusia rajapintoja hallinnollisten tarpeiden täyttämiseen. Lisäksi SCC edellyttää reunaresurssien integrointia osaksi tukiaseman toimintoja. 
Tukiasemaan sidottujen reunaresurssien vuoksi SCC:n rakenne on litteä. 
SCC:n mukaan uudistetun tukiaseman, eli SCeNBce:n, tehtäviin kuuluu reunapalveluille tarkoitetun tietoliikenteen ohjaaminen. Tämä tarkoittaa että SCeNBce monitoroi tietoliikennettä ja ohjaa esimerkiksi kohde IP:n perusteella reunapalvelulle tarkoitetut paketit.

CONCERT on toinen suoran integraation reunajärjestelmä. Ehdotuksen tavoitteena on NFV:n laajamittainen käyttöönotto, jonka seurauksena mobiiliverkon ja reunajärjestelmä voisivat jakaa laskentaresursseja. Järjestelmä rakentuu kolmeen kerrokseen hierarkisesti sijoitettujen resurssien varaan. Nämä resurssit ovat reunalaskennan ja mobiiliverkon toiminnallisuuksien käytettävissä. 
Mobiiliverkon ja reunalaskennan hallinnolliset toimet on yhdistetty Conductor nimiselle entiteetille. Conductor sisältää useita alitoimintoja joiden tehtäviin kuuluu muun muassa resurssien jakaminen kullekkin toiminnallisuudelle. CONCERT ottaa hallinnollisiin tehtäviin ainoastaan yleisellä tasolla ja yksityiskohdat jätetään avoimeksi. Oletettavaa on että CONCERT:ssa mobiiliverkon ja reunajärjestelmän reititys tapahtuu samassa kontekstissa, jolloin järjestelmä ei vaadi erillistä toiminnallisuutta reunasolmuille suuntautuvan tietoliikenteen ohjaamiseen. 

MobiScud ja SMORE edustavat läpinäkyvää integraatiota. SDN-kerros eNodeB ja EPC:n komponenttien välillä mahdollistaa tietoliikenteen monitoroinnin sekä ohjaamisen. Koska varsinaiset reunasolmut eivät ole riippuvaisi mobiiliverkosta, on niiden sijoittelu täysin vapaa.
Koska SDN-kerros sijaitsee eNodeB:n ja EPC:n välillä joudutaan asiakaslaitteen ja reunasolmun välistä liikennettä muokkaamaan. Kyseisellä välillä tietoliikenne on GTP tunneloitua.
Riippuen onko paketti menossa reunasolmulle vai tulossa reunasolmulta, joudutaan tunnelointi purkamaan tai paketoimaan. Tämä saattaa johtaa merkittävään viiveeseen yhdessä monitoroinnin kanssa.

Ainoa epäsuoraa integraatiota edustava ratkaisu on FMC. FMC:n tehtävänä on viedä mobiiliverkon yhteydet nopeammin ulkoverkossa sijaitseville palvelinresursseille. Mobiiliverkon ulkopuolelle sijoitetut resurssit mahdollistavat sen että reunajärjestelmää ylläpitää jokin ulkoinen taho.
FMC ei siis suoranaisesti ota kantaa tapaan jolla reunaresurssit on järjestetty. Mutta on perusteltua olettaa että ne sijaitsisivat mobiiliverkon P-GW:n läheisyydessä.
FMC:n tapa mahdollistaa reunapalvelun ja asiakaslaitteen välinen kommunikaatio perustuu sessiotunnistesiin. Se mahdollistaa asiakaslaitteen ja reunpalvelun IP-osoitteiden vaihtumisen, ilman että viitteet asiakaslaitteen ja reunapalvelun välillä katkeavat. Järjestelmä ei siis edellytä tavallisesta poikkeavaa mekaniikkaa tietoliikenteen reitittämiseksi.
%Reuna-arkkitehtuurien ominaisuuksien vertailu aloitetaan live migraatiosta.
%\par
%Live migraatio on toiminnallisuuksista se joka mahdollistaa suurempien palvelukokonaisuuksien siirtämisen reunajärjestelmässä. Syitä reunalaskennan siirtämiselle voi olla useita ja ehkä yleisin pinnalle noussut syy on asiakaslaitteen liikkuminen verkossa. Muita syitä live migraatiolle voi olla esimerkiksi laskennan siirtäminen enemmän resursseja sisältävälle reunapalvelulle tai ylikuormituksen välttäminen nykyisellä reunasolmulla.
%%migraatio edellyttää reunasolmujen olemista samalla palveluntarjoajalla.
%
%Live migraatio toiminnallisuuden puuttumisen voi tulkita ainakin kahdella tavalla. Joko se tarkoittaa, että reunalaskenta on muodoltaa etälaskentaa, joka on niin hienojakoista, että migraatiolle ei ole tarvetta. Tällöin etälaskenta rajoittuu pieniin työyksiköihin joista odotetaan välitöntä tulosta asiakaslaitteelle. 
%Toinen vaihtoehto on että tarjolla olevat reunasovellukset ovat tyypiltään yleiskäyttöisiä, esimerkiksi pelipalvelimia.
%
%Lähes kaikki reuna-arkkitehtuurit tiedostivat tarpeen live migraatiolle. Ainoastaan SCC ei sisältänyt edes mainintaa reunasovelluksien siirtelystä reunasolmulta toiselle.
%
%
%\par
%Reuna-arkkitehtuurin kommunikaatio
%
%\par
%Reuna-arkkitehtuurin 
%

% Taulukko toiminnoista
\begin{landscape}
    \noindent
\begin{table}[!ht]
\caption{Reunalaskenta-arkkitehtuurien ominaisuudet}
\label{table:features}
\begin{tabularx}{\linewidth}{|c|c|c|p{3cm}|c|p{5cm}|}
\hline 
 \textbf{Ominaisuus} & \textbf{Integraatio} & \textbf{Rakenne} & \textbf{Migraatio} & \textbf{Hallinta} & \textbf{Kommunikaatio} \\ 
\hline 
 FMC & Epäsuora & Vapaa & Palveluiden siirto ulkopuolisten salien välillä
  & FMC Controller & Tavalliset reitityksen, palveluiden ja asiakaslaitteen yhdistämiseen erillinen sessiotunniste \\ 
\hline 
 SMORE & Läpinäkyvä & Vapaa & Ei sisällä & SMORE Controller & SDN monitori ja reititys \\ 
\hline 
MobiScud & Läpinäkyvä & Vapaa & Live migraatio & MobiScud Controller & SDN monitori ja reititys\\ 
\hline 
SCC & Suora & Litteä & Live migraatio & SCM & Monitori ja reititys tukiasemassa \\ 
\hline 
CONCERT & Suora & Hierarkinen & Live migraatio & Conductor & SDN reititys mobiiliverkossa \\ 
\hline 
\end{tabularx} 
\end{table}
\end{landscape}
\begin{table}
	
	
\end{table}
 