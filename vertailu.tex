\subsection{Vertailu}
Seuraavaksi vertaillaan edellä esitettyjä reuna-arkkitehtuuri ehdotuksia niiden ominasuuksien ja rakenteen osalta, sekä analysoidaan ominaisuusjoukon vaikutusta kuhunkin järjestelmään.
Lopuksi arvioidaan reuna-arkkitehtuurien yhdenmukaisuutta ETSI:n MEC spesifikaation kanssa.

Reuna-arkkitehtuurien ominaisuuksien vertailu aloitetaan live migraatiosta.
\par
Live migraatio on toiminnallisuuksista se joka mahdollistaa suurempien palvelukokonaisuuksien siirtämisen reunajärjestelmässä. Syitä reunalaskennan siirtämiselle voi olla useita ja ehkä yleisin pinnalle noussut syy on asiakaslaitteen liikkuminen verkossa. Muita syitä live migraatiolle voi olla esimerkiksi laskennan siirtäminen enemmän resursseja sisältävälle reunapalvelulle tai ylikuormituksen välttäminen nykyisellä reunasolmulla.
%migraatio edellyttää reunasolmujen olemista samalla palveluntarjoajalla.

Live migraatio toiminnallisuuden puuttumisen voi tulkita ainakin kahdella tavalla. Joko se tarkoittaa, että reunalaskenta on muodoltaa etälaskentaa, joka on niin hienojakoista, että migraatiolle ei ole tarvetta. Tällöin etälaskenta rajoittuu pieniin työyksiköihin joista odotetaan välitöntä tulosta asiakaslaitteelle. 
Toinen vaihtoehto on että tarjolla olevat reunasovellukset ovat tyypiltään yleiskäyttöisiä, esimerkiksi pelipalvelimia.

Lähes kaikki reuna-arkkitehtuurit tiedostivat tarpeen live migraatiolle. Ainoastaan SCC ei sisältänyt edes mainintaa reunasovelluksien siirtelystä reunasolmulta toiselle.


\par
Reuna-arkkitehtuurin kommunikaatio

\par
Reuna-arkkitehtuurin 


% Taulukko toiminnoista

\begin{table}
	\caption{Reunalaskenta-arkkitehtuurien ominaisuudet}
	\label{table:features}
	
\end{table}
 