\subsection{Software-defined networking}
Perinteinen tietoliikenneverkko koostuu joukosta erikoistuneista verkkolaitteita. 
Tällaisia laitteita ovat esimerkiksi kytkin, palomuuri ja reititin. Näiden laitteiden joukko muodostaa hajautetun rakenteen, jossa jokainen laite täytyy konfiguroida erikseen.
Laitevalmistajat tarjoavat hallinnointityökaluja, joiden avulla valmistajan omia laitteita on mahdollista konfiguroida suljettua rajapintaa käyttäen. Verkkoinfrastruktuurin hallintaa kuitenkin vaikeuttaa eri laitteiden konfiguraatioiden erilaisuus, sekä puute kokonaisvaltaiselle ohjelmalliselle konfiguroitavuudelle.
Tästä johtuen verkkoon tehtävät muutokset ovat työläitä ja riskialttiita \cite{kreutz2015software}.

Perinteisessä tietoliikennelaitteistossa tiedonvälityskerros ja kontrollikerros ovat tiukasti toisistaan riippuvaisia. Tiedonvälityskerroksella tarkoitetaan itse tietoliikennepakettien välittämiseen käytettävää kerrosta, eli tietoliikennettä ohjaavia laitteistoja. Kontrollikerroksella tarkoitetaan tietoliikenteen reitittämiseksi käytettävää logiikkaa, kuten esimerkiksi reititystauluja. Kontrollikerros on siis tällä hetkellä hajautettuna laitteiston mukana ympäri verkkoa. Tästä johtuen verkko on usein hyvin staattinen ja muutokset kankeita.

SDN (Software-defined networking) eli ohjelmallisesti määritetty verkko on yleistyvä paradigma verkkoympäristöissä.  SDN on ratkaisu, jossa tiedonvälitys- ja kontrollikerros on erotettu toisistaan. SDN:ssä ei ole erikoistunutta verkkolaitteistoa vaan nykyinen tiedonvälitykseen käytetty laitteisto korvattaisiin yleisillä reitittävillä laitteilla\footnote{Reitittämisellä tarkoitetaan tässä yhteydessä pakettien ohjausta ja välitystä yleisessä mielessä}. Kontrollikerroksen toiminnasta vastaa SDN Controller. Se on loogisesti keskitetty entiteetti, joka vastaa näiden reitittävien laitteiden ohjaamisesta.

SDN Controllerin ja reitittävän laitteiston välille oletetaan hyvin määritelty rajapinta, jonka kautta reitittäviä laitteita voidaan hallita \cite{kreutz2015software}. SDN siis toteuttaa \textit{Separation of Concerns} -periaatetta jakamalla verkon reititysmäärittelyjen konfiguroinnin ja itse laitteistopohjaisen toteutuksen omiin osiinsa. SDN Controller tarjoaa rajapinnan ylöspäin ohjelmalliselle verkkokonfiguroinnille ja hoitaa sääntöjen tulkkaamisen alaspäin. OpenFlow on yksi tunnetuimmista SDN Controllerin ja verkkolaitteiden välisestä protokollasta\footnote{\url{https://www.opennetworking.org/software-defined-standards/specifications/}}.

% TODO Flow sääntöjen toiminta.

Tarve SDN pohjaisille ratkaisuille kumpuaa jo aiemmin mainitusta konfiguraation työläydestä ja dynaamisuuden tarpeesta. Esimerkiksi virtuaalikoneiden käytön yleistyessä tarve verkon ohjelmalliselle hallittavuudelle on kasvanut \cite{jammal2014software}. Virtuaalikoneiden sijainti verkossa ei välttämättä ole kiinteä, vaan virtuaalikone saattaa siirtyä esimerkiksi migratoinnin seurauksena. Virtuaalikoneita saattaa tulla ja poistua verkosta. Perinteisessä verkkoympäristössä esimerkiksi MAC osoitetaulujen päivittäminen saattaa aiheuttaa yhteyskatkoksia palvelimeen \cite{jammal2014software}. SDN pohjaisia ratkaisuja on jo olemassa, mutta niiden käyttö ei vielä ole korvannut perinteisiä verkkolaitteita. 

