\section{Yhteenveto} \label{yhteenveto}
Reunalaskennan tarjoaminen mobiiliverkon kautta vaatii kahden eri järjestelmän yhteistoiminnallisuutta.
Olemassa oleva mobiiliverkko on hajautettu ja suljettu järjestelmä.
Tämän seurauksena siihen tehtävien muutoksien tulee olla tarkkaan harkittuja.
Tässä tutkielmassa esiteltiin viisi erilaista reuna-arkkitehtuuriehdotusta, joista jokainen sisälsi uniikin joukon reunajärjestelmän ominaisuuksia.

Tämä tutkielma identifioi reuna-arkkitehtuuriehdotuksista viisi ominaisuutta.
Nämä ominaisuudet edustavat jonkin suunnittelupäätöksen vaikutusta johonkin reunalaskennan mahdollistavaan toimintoon tai toimijaan.
Tunnistetut ominaisuudet olivat live migraatio, integraation tyyppi, kommunikaatio, hallinta ja rakenne.
Rakenne oli muista ominaisuuksista poikkeava, koska se määräytyy implisiittisesti muiden arkkitehtuurissa tehtyjen päätöksien pohjalta.

Ominaisuuksien joukossa ilmeni selkeitä riippuvuksia ja riippuvuuksien perusteella keskeisimmäksi arkkitehtuurissa tehtäväksi päätökseksi todettiin integraation tyyppi.
Reunajärjestelmään toteuttamiseen käytettävällä integraation tyypillä on keskeinen vaikutus kaikkien muiden ominaisuuksien toteuttamiseen.
Etenkin reunajärjestelmää toteutettaessa integraation tyyppi vaikuttaa siihen, kuinka paljon olemassa olevaa mobiiliverkkoa voidaan hyödyntää tai paljonko sitä tulee uusia. 

Tukielman lopussa käsiteltiin reuna-arkkitehtuuriehdotuksien yhteensopivuutta ETSI MEC spesifikaation vaatimuksiin.
Esitetyt arkkitehtuurit olivat pääsääntöisesti yhteensopivia, mutta parhaimmat lähtökohdat vaikuttivat olevan läpinäkyvää integraatiota edustavilla arkkitehtuureilla.
Jokaisessa arkkitehtuurissa kuitenkin on omat kompromissinsa ja ehdotettujen arkkitehtuurien vertailu vaatisi yhdenmukaistettuja testejä. 
Koska reunalaskennan keskeisenä muuttujana on asiakaslaitteen ja reunapalvelun välinen viive, ei pelkät ohjelmalliset simulaatiot välttämättä riitä antamaan tarpeeksi tarkkaa kuvaa järjestelmän toiminnasta.
Etenkin NFV toiminnallisuuksiin nojautuvien ratkaisujen osalta kysymykseen tulee NFV:llä toteutettujen toimintojen suorituskyky.

%
%Ominaisuuksien esittelyn jälkeen käsiteltiin varsinaiset reuna-arkkitehtuurit sekä ETSI MEC spesifikaatio.
%Näiden tietojen pohjalta suoritettiin ominaisuuksiin perustuvaa vertailua ja tarkasteltiin ehdotettujen reuna-arkkitehtuurien yhteensopivuutta ETSI MEC spesifikaatiossa esitettyihin vaatimuksiin.



 
%
%Tässä tutkielmassa esiteltiin mobiiliverkkoympäristöön integroitavia reunalaskenta-arkkitehtuuriehdotuksia. 
%Luvussa \ref{perusteet} käsiteltiin reunalaskennan tavoitteet yleisellä tasolla ja määriteltiin reunalaskennan kannalta keskeiset käsitteet.
%Koska reunalaskenta on vielä kehitysvaiheessa, sen määritelmä elää edelleen. Tässä tutkielmassa esitetty reunalaskennan terminologia ja määritelmät pohjautuvat ETSI MEC spesifikaatiossa käytettyyn. 
%Käsitteiden yhteydessä esiteltiin mobiiliverkkoon liitettävän reunajärjestelmän toimintaympäristö.
%
%Tämän jälkeen luvussa \ref{mahd} esiteltiin reunajärjestelmän toteuttamiseksi ehdotettujen teknologioiden joukko.
%Mahdollistavien teknologioiden joukko perustui tutkielman puitteissa käsiteltyjen arkkitehtuuriehdotuksien yhteydessä ilmenneisiin teknologisiin tarpeisiin.
%
%Luvussa \ref{ominaisuudet} esiteltiin reuna-arkkitehtuureista tunnistetut ominaisuudet.   
%
%Luvussa \ref{ratkaisut} käsiteltiin reunalaskenta-arkkitehtuuriehdotukset sekä lopuksi käsiteltiin niiden yhtenevyyttä ETSI:n MEC spesifikaation kanssa.
