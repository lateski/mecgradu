

%Tähän voisi laittaa kuvan komponenttien asemmoitumisesta toisiinsa.
\subsubsection{Asiakaslaite}
Asiakaslaite (user equipment) on yleisnimitys laitteelle, joka on reunan tai pilven tarjoamien palveluiden asiakas. 
Asiakaslaitteen tyyppiä ei ole rajattu ja asiakaslaite voi olla esimerkiksi älypuhelin, älylasit tai vaikka verkkoyhteydellä varustettu auto. 
Yhdistävänä tekijänä on siis jonkinlainen yhteys reuna- tai pilvipalveluihin. Yleisesti yhteys on tyypiltään langaton. 
Tässä tutkielmassa asiakaslaitteella tarkoitetaan älypuhelinta, jollei toisin mainita.

Verkkohierarkian näkökulmasta asiakaslaitteet ovat lehtisolmuja. Tämä tarkoittaa että asiakaslaitteet toimivat ainoastaan palveluiden kuluttajina eivätkä siis tarjoa itse palveluita. Myöskään kulutettavien palveluiden tyypillä ei ole juurikaan merkitystä, kun asiaa käsitellään infrastruktuuri/arkkitehtuuritasolla.

Esimerkkinä asiakaslaitteen toiminnasta reunapalvelua hyödyntäen voisi olla ajoneuvo, joka käyttää reunapalveluita tiedon välittämiseen muille lähistöllä oleville ajoneuvoille.


%Asiakaslaite on yleisnimitys laitteelle, joka hyödyntää reunan tai pilven tarjoamia palveluita jonkin tietoliikenneyhteyden avulla.
%Reunalaskentaa käsittelevässä kirjallisuudessa asikaslaitteeseen viitataan usein UE (User Equipment) termillä. 
%Se mitä konkreettista laitetta asiakaslaitteella tarkoitetaan riippuu kontekstista. 
%Tässä tutkielmassa asiakaslaitteella tarkoitetaan älypuhelinta jollei toisin mainita. 
%Esimerkkinä jostain toisesta asiakaslaitteesta on auto, joka on varustettu mobiiliverkkoyhteydellä. Auto voi käyttää reunapalveluita esimerkiksi kommunikoidakseen muiden lähistöllä olevien autojen kanssa.
%Asiakaslaitteet verkkoyhteyksien muodostaman hierarkian "lehtisolmuja" joka tarkoittaa että ne eivät enää tarjoa palveluitaan muille asiakaslaitteille. 


\subsubsection{Reuna}
Reuna koostuu useista toiminnallisista entiteeteistä, jotka voidaan jakaa sekä fyysisiin, että loogiisin entiteetteihin. On kuitenkin hyvä ymmärtää koko reunan käsittävä reuna-alue, joka esitellään ensimmäisenä. Tämän jälkeen esitellään reunasolmu, joka on reunajärjestelmän keskeisin fyysinen rakennuspalanen. Lopuksi esitellään pääasiassa ohjelmallisen tason toimivat toimijat reuna-alusta ja reunasovellus.

\paragraph{Reuna-alue} 
Reuna-alue tai reuna ei ole mikään tarkasti rajattu alue, vaan reunalla usein tarkoitetaan jonkin tietyn kontekstin mukaista reunaa. 
Yleisesti reunalla voidaan viitataan alueeseen, joka ulottuu asiakaslaitteelta runkoverkkoon asti. Reunalle ei siis ole tarkkaa määritelmää.
Reuna-alue rajautuu siinä esiintyvien toimijoiden mukaan jollekkin välille. 
Esimerkiksi mobiiliverkon tukiasemien yhteyteen rakennettua reunajärjestelmää käsiteltäessä, reunalla tarkoitetaan ainoastaan reunajärjestelmän asiakaslaitteita palvelevien osien muodostamaa vyöhykettä. 
Yleisenä nyrkkisääntönä voidaan pitää verkkoyhteyksien viivettä, suhteessa muuhun internettiin, koska reuna-alueella viiveiden tulisi siis olla muuta internettiä nopeampia. Toisin sanoen palveluiden tulisi sijaita lähempänä.


\paragraph{Reunasolmu} 
Tämän tutkielman kontekstissa reunasolmulla (mobile edge host) tarkoitetaan yksittäistä reunalaskentaa suorittavaa entiteettiä. Reunasolmu voi koostua esimerkiksi mobiilitukiaseman ja palvelinlaitteiston muodostamasta kokonaisuudesta. Reunasolmu sisältää vähintään reunasovelluksien suorittamiseen tarvittavan laitteiston, sekä toiminnallisesti reunasovelluksien suorittamiseen tarvittavat hallinnolliset toimet. 
Kuten aiemmin mainittu, reunajärjestelmä koostuu joukosta reunasolmuja, jotka on hajautettu maantieteellisesti.
Reunasolmut siis eroavat toisistaan vähintään sijainnin perusteella, mutta voivat erota myös käytettävissä olevien laskenta ja tallennus resurssien osalta.
Reunasolmun sijaintiin vaikuttaa käytössä oleva reuna-arkkitehtuuri.
%Kuten juuri mainittu, reunasolmujen sijaintiin vaikuttaa käytettävä reuna-arkkitehtuuri. 
Teoriassa reunasolmu voi sijainta asiakaslaitteesta \textit{yhden hypyn päässä}, jolloin asiakaslaitteella olisi suora yhteys reunasolmuun, mutta on myös mahdollista, että reunasolmu sijaitsee kauempana esimerkiksi runkoverkon reunalla. 
Laskentaresurssien osalta reunasolmu voi olla mitä tahansa vähäisillä laskenta ja tallennus resursseilla varustetun WiFi-tukiaseman ja kokonaisen palvelinklusterin väliltä. 

\paragraph{Reuna-alusta}
(Edge platform) on ohjelmistotason toimija, joka tarjoaa rajapinnan reunasovelluksien suorittamista varten. Toisin sanoen siis tarjoaa reunasovelluksille toimintaympäristön.
Reuna-alustan tehtävät eivät rajaudu ainoastaan yksittäiseen reunasolmuun, vaan sen lisäksi se hoitaa hallinnollisia tehtäviä kuten tietoliikenteen ohjausta. Lisäksi reuna-alustan tehtäviin voidaan lukea reunasovelluksia suorittaviin virtuaalikoneisiin liittyvät hallinnolliset toimet. Esimerkiksi myöhemmin kappaleessa \ref{livemigraatio} esiteltävä virtuaalikoneiden migraatio reunasolmulta toiselle on reuna-alustan vastuulla.
Reunasovelluksien lisäksi reuna-alusta voi itsessään tarjota jonkinlaista toiminnallisuutta, joka ei suoranaisesti ole reunasovellus vaan esimerkiksi kommunikaatioväylä laitteelta-laitteelle (machine-to-machine) esimerkiksi ajoneuvojen väliselle viestinnälle. 

\paragraph{Reunasovellus}
(Edge application) on yksittäinen reunasolmulla suoritettava ohjelmisto, jonka kuluttajana voi toimia asiakaslaite tai toinen reunasovellus. Reunasovellus ei ota kantaa millaista palvelua sillä tuotetaan ja sen rajoitteet asettaa pääasiassa käytössä oleva reuna-alusta. Reunasovelluksen tuottama palvelu voi olla käyttäjäkohtainen, esimerkiksi virtuaalikone johon käyttäjä voi ottaa etäyhteyden tai tukevaa laskentaa, esimerkiksi kasvojen tunnistusta suorittava palvelinohjelma. Esimerkki ei-käyttäjäkohtaisesta reunasovelluksesta olisi pelipalvelin, jota voivat käyttää reunasolmun lähistöllä olevat pelaajat. 

\subsubsection{Pilvi}
Pilvellä tarkoitetaan internetin ytimen läheisyydessä sijaitsevaa aluetta.
Pilven voidaan ajatella laajenevan kohti reunaa ja olemaan sen kanssa myös osittain limittäin.
Tavallisessa asiakas-palvelin -mallissa palvelun tuottavan palvelimen voidaan ajatella sijaitsevan pilvessä. Kun mukaan otetaan reunajärjestelmä, palveluiden tuottaminen hajautuu reunan ja pilven kesken.


%Asiakaskohtaiselle reunainstanssille ei ole mitään vakiintunutta nimeä.
%Cloudlet on yksi ehdotettu toteutustekniikka tällaiselle asiakaskohtaiselle reunalla sijaitsevalle virtuaali-instanssille \cite{satya09}.

\subsection{Mobiiliverkko}
\begin{figure}[htb]
\includegraphics[scale=0.5]{EUTRAN}
\caption{Mobiiliverkon rakenne} \label{ekakuva}
\end{figure}

MEC ehdotuksia tarkasteltaessa, olemassa olevien mobiiliverkkojen arkkitehtuurit ovat keskeisessä asemassa. Useat reunalaskentaa käsittelevät ehdotukset on tarkoitettu integroitaviksi osaksi olemassa olevaa mobiiliverkkoarkkitehtuuria. 
Erityisesti asiakaslaitteiden ollessa mobiililaitteita ja mahdollisimman pieniä viiveitä tavoiteltaessa, reunalaskenta ratkaisujen integroiminen osaksi mobiiliverkkoa on väistämätöntä. Seuraavaksi käydään pääpiirteittäin läpi 4G mobiiliverkon arkkitehtuurin funktionaaliset osat.

Yksinkertaistettuna mobiiliverkko koostuu kahdesta osasta: radiorajapinnasta ja mobiilinverkon runkoverkosta (puhelinkeskuksesta?). 3GPP kehittämässä LTE standardissa radiorajapinnan sisältävä osuus on nimeltään E-UTRAN (Evolved UMTS Terrestrial Radio Access Network) ja runkoverkon osuus EPC (Evolved Packet Core).

E-UTRAN tehtävänä on toimia rajapintana asiakaslaitteen ja EPC:n välillä. Asiakaslaitteiden suuntaan yhteys on radiosignalointia ja yhtenä E-UTRAN tehtävänä onkin radioresurssien hallinointi. 
E-UTRAN sisältää verkon puolella pääasiallisena toimijana eNodeB tyyppisiä tukiasemia \cite{etsieutran}, myös muutamia poikkeustapauksia on, mutta ne jätetään käsittelemättä.
Tukiasema on asiakaslaitetta lähimpänä sijaitseva funktionaalinen verkon osa ja sen seurauksena se on houkutteleva kohde MEC ratkaisuille. Tukiasemaa voikin ajatella \textit{reunan} viimeisenä etappina ennen asiakaslaitetta. Tämä tutkielma käsittelee pääasiassa ratkaisuja, jotka keskittyvät LTE verkkoihin kohdistuvia ratkaisuja, joten tukiasemista puhuttaessa tarkoitetaan nimenomaan E-UTRAN mukaista eNodeB:tä. 

ENodeB:n tehtävänä on kommunikoida radioyhteyttä käyttäen asiakaslaitteen kanssa ja välittää molemminsuuntaista liikennettä EPC:n (Evolved packet core) suuntaan S1 rajapinnan kautta. eNodeB:t ovat myös yhteydessä toisiinsa X2 rajapintaa käyttäen. X2 rajapintaa käytetään tukiasemien väliseen kommunikointiin, joka sisältää handoverin yhteydessä tehtävää asiakaskontekstin siirtoa ja erinäisiä muita hallinnollisia toimintoja.  
Handoverin vaikutukset näkyvät myös MEC puolelle ja handoveria toimintaa käsitellään tarkemmin kappaleessa XYZ.

Mobiiliverkkojen tietoliikenne koostuu kahdesta eri tasosta: kontrollikerroksesta ja datakerroksesta.
Kontrollikerroksella on hallinnointiin liittyviä protokollia joiden tehtävänä on välittää esimerkiksi asiakaslaitteeseen liittyviä kontekstitietoja entiteetiltä toiselle. Datakerros puolestaan välittää IP liikennettä asiakkaan ja palveluiden välillä. 
EPC koostuu kolmesta alikomponentista jotka ovat MME (Mobility Management Entity), SGW (Serving Gateway) ja PGW (PDN gateway) \cite{etsilte}.
SGW:n eli palveluyhdyskäytävän tehtävänä on välittää datakerroksen liikennettä asiakaslaitteen ja ulkoisten palveluiden välillä. SGW on yhdistettynä PGW:hen, jonka tehtävänä on toimia IP tason reitittimenä EPC:n ja ulkoisen verkon välillä. PGW toimii asiakkaan ulkoverkon yhteyksien kiintopisteenä \cite{3gppepc}. MME on EPC:n hallinnollinen komponentti ja se toimii ainoastaan kontrollikerroksessa.

\subsection{Reuna-arkkitehtuuri}
%Määrittele mikä on reuna-arkkitehtuuri
Tämän tutkielman puitteissa arkkitehtuurilla tarkoitetaan reuna-alustan ja reunasolmujen muodostamaa järjestelmää. Reuna-alustan toiminnallisuuksia voisi eritellä vielä tarkemmin, etenkin reunasovelluksien hallinnan osalta. Cloudlet on konkreettinen toteutus reuna-alustan reunasovelluksia hallinnoivasta osasta, mutta se ei kata kuin yhden reunasolmun kerrallaan. Tämän lisäksi olemassa on reunasolmujen välisestä hallinnasta vastaava kerros, tosin tämä kerros voi olla ulkoistettu erillisille hallinnolliselle toimijalle, jolloin hallinnollinen osa reuna-alustasta on varsinaisten reunasolmujen ulkopuolella. 

Reunasovelluksien toimintaa ei käsitellä tarkemmin. Taxonomy paperissa on hyvä jaottelu erilaisista sovelluksista. Tämä aihepiiri laajenee sovelluksien jakamiseen osittain reunalla suoritettaviin (offloading) tai mobiilisovelluksen suoritusta tukeviin sovelluksiin. 

