\section{Rakenne}
Seuraavaksi käydään läpi reunan rakennetta. Koska aihepiirinä on arkkitehtuurit, varsinaisten toteutettujen reunajärjestelmien rakennetta voidaan käydä ainoastaan hypoteettisesti. 
Eli mitä arkkitehtuuri mahdollistaa tai rajoittaa.
Ensimmäiseksi käsitellään reuna-arkkitehtuurien vaikutuksen toteutettavaan järjestelmään.
Tämän jälkeen esitellään reunan fyysisen rakenteen eri mallit, jossa keskeisessä osassa ovat reunasolmujen sijainnit.
Lopuksi käydään läpi reunan rakenteen merkitystä järjestelmän toiminnalle.

\subsection{Arkkitehtuurin vaikutus rakenteeseen}
%hallinnon sijainti
%integraatio mobiiliverkkoon
%hallinnoivan järjestelmän omistajuus.

\subsection{Rakennetyypit}
Kirjoita meta

\subsubsection{Litteä rakenne}
Litteällä rakenteella tarkoitetaan että jokainen reunasolmu on hierarkisesti samalla tasolla toisiinsa nähden. Litteän järjestelmän keskeisin päätös on valita taso, jolle reunasolmut sijoitetaan. (Voisiko sanoa että tällainen järjestelmä on reunavyöhyke)

Yksinkertainen esimerkki tällaisesta toteutuksesta olisi reunajärjestelmä, jossa reunasolmut sijoitettaisiin mobiiliverkon tukiasemien yhteyteen. Esimerkin järjestelmä  toteuttaisi äärimmäistä hajautusta ja lisäksi se olisi fyysisesti erittäin lähellä asiakaslaitetta. Tällaisessa rakenteessa kuitenkin on ilmeisiä ongelmia. Järjestelmän käyttö olisi teoriassa mahdollista ainaostaan niiden tukiasemien ympäristössä, joissa reunajärjestelmä sijaitsee \cite{hassuautokuva}. Tämän seurauksena reunalaskennan laajamittainen käyttöönotto olisi riippuvainen tukiasemiin sijoitettavien reunasolmujen hankinnasta.

Vaihtoehtoinen ratkaisu olisi siirtää reunasolmuja kauemmaksi tukiasemista. Esimerkiksi siten että yksi reunasolmu vastaisikin useamman tukiaseman kautta tulevasta reunalaskennasta. Mobiiliverkko kontekstissa, reunasolmu sijaitsisi siis joko tukiasemien ja EPC:n välillä tai vasta EPC:n takana.  
Useamman tukiaseman niputtamista yhden reunasolmun vastuulle kutsutaan klusteroinniksi. 
Keskeisenä haasteena klusteroinnissa on valita oikean kokoiset klusterit ja sijoittaa oikea määrä resursseja kuhunkin klusteriin.
Reunasolmujen siirtäminen kauemmaksi asiakaslaitteista pitäisi siis mahdollistaa helpompi ja edullisempi käyttöönotto, koska tarvitaan suhteessa vähemmän reunasolmuja ja reunasolmujen ylläpito on näin ollen edullisempaa.
Kompromissina on kuitenkin suurempi viive reunasolmujen ja asiakaslaitteiden välillä. 
Onkin siis tärkeää että reunasolmuja ei viedä liian kauas reunasta, jolloin palvelun laatu heikkenee \cite{mao17}. 

Kuitenkin riippumatta litteän rakenteen sijainnista, sen pohjimmaisena ongelmana on resurssien kiinteys. Ympäristössä jossa reunalaskennan määrä vaihtelee, seuraa tilanne jossa reunasolmu on joko ylikuormitettuna tai alikuormitettuna suhteessa käytössä oleviin resursseihin \cite{tong2016hierarchical}. Jotta reunasolmut pystyisivät suoriutumaan kaikesta reunalaskennasta, reunasolmujen resurssit jouduttaisiin mitoittamaan rasitus-huippujen mukaan. Tästä seuraa lähes jatkuvaa alikuormitusta, joka tarkoittaa että reunajärjestelmän kustannukset olisivat korkeammat kuin olisi tarpeen. 
Tämän ongleman ratkaisuksi on esitetty hierarkista järjestelmää.

\paragraph{kirjoittajan huomioita}
\begin{list}{•}{•}
\item \cite{malandrino2016close} esittävät tosielämän klusterointia joka pyrkii yhdistämään sekä viiveen että resurssien utilisaatiota. 
\item \cite{mao17} kertoo että lähelle tuotu laskenta on kalliimpaa johtuen siitä että on todennäköistä että siellä missä laskentaa tarvitaan eniten, on myös paljon ihmisiä, joka myös näkyy esim tilavuokrissa.
\item Lisäksi Mao et al esittävät että kauemmaksi sijoitetut palvelimet laskevat palvelun laatua. 
\item  Itseasiassa kauas viedyt hajautetut rakenteen alkavat muistuttaa fog tyyppistä palvelua
\item In 2012, Flavio Bonomi and his colleagues introduced the term fog computing to refer to this dispersed cloud infrastructure. F. Bonomi et al., “Fog Computing and
Its Role in the Internet of Things,”
Proc. 1st Edition MCC Workshop
Mobile Cloud Computing (MCC 12),
\end{list}

\subsubsection{Hierarkinen rakenne}
Hierarkinen rakenne on parannusehdostus reunalaskennan oletetulle litteälle rakenteelle \cite{tong2016hierarchial}
Reunasolmujen hierarkisella asettelulla tarkoitetaan järjestelmää, jossa reunasolmut on jaettu kerroksiin. Alimmalla kerroksella sijaitsevat reunasolmut ovat lähimpänä asiakaslaitetta, mutta niiden sisältämät laskentaresurssit ovat vähäisiä. 
Ideana on että alemmalla kerroksessa sijaitsevat reunasolmut voivat siirtää laskentaa ylemmällä kerroksella sijaitsevalle reunasolmulle, jolla on enemmän resursseja. Tasojen määrälle ei ole mitään rajaa, mutta useimmat ehdotukset sisältävät kaksi tai kolme tasoa.
Ylemmällä tasolla sijaitsevalla reunasolmulla on vastuullaan useampi alemman tason reunasolmu. Hierarkinen rakenne on siis puun mallinen.

Hierarkisen rakenteen keskeisenä tavoitteena on reunajärjestelmän resurssien käyttöasteen parantaminen \cite{tong2016hierarchial}. Ylemmillä kerroksilla sijaitsevat resurssit ovat suuremman joukon käytössä, hieman kuten litteässä mallissa tukiasemia klusteroitaessa. 
Hierarkisessa rakenteessa on kuitenkin riskinsä. Järjestelmän voidaan olettaa monimutkaistuvan jos laskentaa tehdään useassa kerroksessa. Ja etenkin järjestelmän ylempien kerroksien etäisyys asiakaslaitteisiin kasvaa, joka voi potentiaalisesti johtaa viiveiden kasvuun ja palvelun laadun heikkenemiseen.


\subsubsection{Vertailu}
>johdanto vertailuun

Tong et al \cite{tong2016hierarchical} esittävät tukimuksessaan vertailua litteälle ja erilaisille hierarkisille rakenteille. Tilanteessa jossa järjestelmään on jaettavissa kiinteä määrä resursseja, kolmeen kerrokseen jaettu järjestelmä vaikutti optimaalisimmalta.
Vertailun mittarina käytettiin aikaa joka reunapalvelulla kului vastata annettuihin laskennallisiin tehtäviin. Kolmikerrokisen mallin etuina oli laskentaresurssien määrä ylemmillä kerroksilla, jolloin etenkin ruuhkaisissa tilanteissa se suoriutui tehtävistään nopeammin. Lisäksi todettiin että ylemmillä tasoilla sijaitseva tehokkaampi laskenta pysty kompensoimaan siirtämisestä aiheutuvaa viivettä. 

Hierarkinen rakenne vaatii vielä tutkimusta. Etenkin tilanteissa joissa viiveet kerroksien välillä ja siirrettävän tiedon määrä on suurempi.




















