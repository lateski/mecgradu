\subsection{Hallinta}

%Reunalaskenta-arkkitehtuurissa hallinnalla tarkoitetaan sellaisia toimia, jotka mahdollistavat, ylläpitävät ja säätelevät asiakaslaitteen ja reunasolmun välistä kommunikaatiota. Hallinnosta vastaavana entiteetin tai entiteettien keskeinen tehtävä on erilaisiin tapahtumiin reagoiminen. Yleisimmin arkkitehtuurien yhteydessä käsiteltävät tapahtumat ovat reunalaskennan aloittaminen, laskennan migraatio ja kommunikaatioväylän avaaminen. Tämän lisäksi hallinnollisiin toimiin oletetaan reunalaskennan resurssien varaaminen sekä niiden optimointi. Näitä ei kuitenkaan korkealla arkkitehtuuritasolla kuvata.


%%%%%%%%%%%%%%%%%%%%%%%

%Reunan hallinta
%Mitä tarkoitetaan hallinnalla?
Reunajärjestelmän hallinta koostuu joukosta reunajärjestelmän toimintaa sääteleviä entiteettejä. 
Tutkielmassa käsiteltävistä reunalaskenta-arkkitehtuureista identifioitiin päävastuussa oleva hallinnollinen toimija sekä joitakin hallinnollisia toimintoja.
Edellä kuvattu live migraation laukaiseminen ja kommunikaatioväylän muodostaminen olivat keskeisimmät tunnistetut hallinnolliset toiminnot.
Näiden toimien perusperiaate on, että hallinnollinen toimija saa herätteen jostain mobiiliverkon tapahtumasta ja välittää tarvittavat toimenpiteet reunasolmuilla sijaitsevalle reuna-alustalle. 

%Reunan hallinta koostuu joukosta hallinnollisia toimia toteuttavia entiteettejä.
%Tämän tutkielman yhteydessä hallinnolliset toimet rajautuvat pääasiassa niihin toimintoihin, jotka mahdollistavat, ylläpitävät tai säätelevät asiakaslaitteen ja reunasolmun välistä kommunikaatiota. 
%Arkkitehtuuritasolla keskeisimmät hallinnolliset toimijat liittyvätkin juuri reunasolmujen saavutettavuuden varmistamiseen.
%Tämän tutkielman esittelyjen ulkopuolelle jäävät siis ne hallinnolliset toimijat, jotka esiintyvät arkkitehtuurikuvauksissa pääasiassa mainintana. Esimerkiksi virtuaalikoneisiin pohjautuvien järjestelmien resurssien oikeanlainen utilisointi \cite{yousaf16fine, taleb2017multi} on keskeinen haaste reunalaskennassa, mutta arkkitehtuuritasolla siihen ei oteta juurikaan kantaa ja sen suunnittelu jätetään järjestelmän toteuttajan vastuulle.


Pääasiassa hallinnollisten toimijoiden rooli esiintyy reunalaskenta-arkkitehtuurissa passiivisena tarkkailijana, jonka tehtävä on reagoida mobiiliverkon tapahtumiin. 
Esimerkki tällaisesta tapahtumasta on mobiiliverkossa tapahtuva handover, johon hallinnollinen entiteetti saattaa haluta reagoida esimerkiksi aloittamalla reunalaskennan migraation lähemmäksi käyttäjää. 
Myös aktiivisten hallinnollisten toimijoiden tarve tiedostetaan, mutta tällä arkkitehtuuritasolla niitä ei määritellä.
Aktiivisten hallinnollisten suunnittelu jätetään reunajärjestelmän toteuttajalle.
Aktiivisia toimia ovat muun muassa reuna-alustalla suoritettavien virtuaalikoneiden elinkaaren hallinta \cite{yousaf16fine}. Kappaleessa \ref{cloudlet} esitellään Cloudlet niminen ratkaisu, joka kuvaa virtuaalikoneiden elinkaaren hallintaa ja ehdottaa live migraatio toiminnallisuutta reunasolmujen välille.
%Mihin tässä tutkielmassa ei paneuduta?

%Arkkitehtuurien esittelyiden yhteydessä hallinnosta vastaava entieetti esitellään kaikissa arkkitehtuureissa. Hallinnosta vastaavat entiteetit arkkitehtuureittain ovat SCM (Small Cell Cloud Manager) Small Cell Cloudissa \cite{lobillo15scc, gambetti15dist}, conductor CONCERT:ssa \cite{liu2014concert}, FMC Controller FMC:ssä \cite{taleb2013follow}, MC (MobiScud Controller) MobiScud:ssa \cite{wang2015mobiscud}, ja edge orchestrator ETSI MEC refenssiarkkitehtuurissa \cite{etsirefarch}. \textit{Tarkista onko kaikki}

%Hajautettu & Keskitetty
Hallinnollinen vastuu voi olla reunajärjestelmässä hajautettuna tai keskitettynä.
Hajautetussa mallissa hallinnollinen entiteetti vastaa jostain reunasolmujen osajoukosta \cite{lobillo15scc}. Hajautus voi myös olla hierarkinen jolloin yhdellä hallinnollisella entiteetillä on ylemmän tason hallinnollinen toimija joka tarpeen tullen vastaa reunalaskennan organisoinnista \cite{mach17mobile}.

Kuten muutkin ominaisuudet, myös hallinnollisen toimijan sijainti riippuu reunajärjestelmän integraation tyypistä. 
Suorassa integraatiossa hallinnollinen toimija on mahdollista sijoittaa käytännössä mihin tahansa \cite{mach17mobile}.
Läpinäkyvässä integraatiossa hallinnollinen toimija sijaitsee yleensä lähellä monitoritoiminnallisuutta, mutta kuitenkin mobiiliverkon ulkopuolella. 
Epäsuorassa integraatiossa hallinnolliset toimijat sijaitsevat mobiiliverkon ulkopuolella.

