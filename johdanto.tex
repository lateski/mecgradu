
\section{Johdanto}
Reunalaskennan (Multi-access edge computing, MEC) tavoittena on tuoda laskentaresursseja lähemmäksi asiakasta.
Verkkoyhteyksistä puhuttaessa läheisyyden merkitys laajenee loogiseen läheisyyteen, jonka mittarina toimii verkkoyhteyden viive asiakkaan ja palvelun välillä \cite{satyanarayanan2017emergence}.
Asiakkaan lähelle sijoitetut laskentaresurssit mahdollistavat palveluiden tuottamisen pienemmällä viiveellä ja vakaammilla verkkoyhteyksillä.

%Se tarjotaa laskentaresursseja ja muita palveluita asiakaslaitteille pienellä viiveellä.
%Reunalaskennan kohderyhmänä puhutaan pääasiassa laitteista joiden resurssit ovat rajalliset.
%Suuren osan tästä kohderyhmästä muodostavat mobiiliverkossa toimivat mobiililaitteet kuten älypuhelimet. Nämä laitteet ovat perinteisesti


Suuren osan reunalaskennan kohderyhmästä muodostavat mobiililaitteet kuten älypuhelimet ja kannettavat. 
Mobiililaite käyttää verkkoyhteyksiinsä langattomia verkkoja ja on riippuvainen akkuvirrasta.
Mobiililaskennalla (Mobile Computing) tarkoitetaan kannettavilla laitteilla suoritettevaa ohjelmistojen suoritusta. 
Mobiililaskenta on rajoittuneempaa kuin pöytätietokoneella suoritettava laskenta. Mobiililaskentaa rajoittaa mobiililaitteen laskentaresurssien määrä, verkkoyhteyden laadun vaihtelu, mobiililaitteen kestävyys ja käytettävissä olevan energian määrä \cite{satya96}. 
Mobiililaitteiden valmistajat joutuvat tasapainottamaan laskentaresurssien määrää ja virrankulutusta yhdessä laitteen koon kanssa. Lisäksi akkuvirran rajallisuus johtaa kompromisseihin, jotka näkyvät mobiililaitteen suorituskyvyssä\cite{satya01pervasive}.

Mobiililaitteiden suorituskykyyn liittyvien kasvavien odotuksien seurauksena, laskennan siirtäminen mobiililaitteilta muualle suoritettavaksi on nostanut reunalaskennan varteenotettavaksi ratkaisuvaihtoehdoksi. Reunalaskennan avulla mobiilaitteiden on mahdollista siirtää resurssi-intensiivistä laskentaa lähistöllä sijaitsevalle reunalaskennasta vastaavalle yksikölle. 
Laskennan siirtämisen motiivina voi toimia esimerkiksi akkuvirran säästäminen tai suoritusajan lyhentäminen. Tämä tulee kuitenkin laskennan siirtoon kuluvan ajan ja energian kustannuksella. 

Perinteiset verkkopalvelut sijaitsevat keskitetyissä palvelinsaleissa – pilvessä.
Koska pilvi tarjoaa näennäisesti äärettömästi laskentaresursseja, se saattaa kuulostaa houkuttelevalta vaihtoehdolta siirtää laskentaa pilveen suoritettavaksi.
Asiakkaasta kaukana sijaitseva pilvi sisältää muutamia ongelmia.
Asiakkaan ja pilven välinen yhteys sisältää merkittävän määrän viivettä \cite{satya09}, jolloin se ei sovellu reaaliaikavaatimuksen sisältävien palveluiden suorittamiseen.
WAN verkon viiveeseen ei juurikaan enää voida vaikuttaa, vaan ainoana vaihtoehtona on laskentaresurssien tuominen lähemmäksi asiakasta \cite{satya09}.
Lisäksi tietoliikenteen määrän kasvun seurauksena, kaistanleveys internetin läpi kulkevan tietoliikenteen osalta ei välttämättä täytä vaatimuksia. 
Esimerkki reaaliaikaisesti tehtävä kasvojen tunnistus mobiililaitteen välittämästä videovirrasta (video stream) vaatii kaistanleveyttä sekä pientä viivettä.

Reunalaskentaa voi hyödyntää useaan eri käyttötarkoitukseen.
Yhtenä reunalaskennan palvelumuotona on edellä kuvattu mahdollisuus siirtää resurssi-intensiivinen laskenta nopean yhteyden päässä sijaitsevalle palvelinlaitteistolle.
Reunalaskenta myös mahdollistaa palveluiden tuottamisen lähempänä asiakasta. Tämä sisältää esimerkiksi mahdollisuuden hyödyntää reunalla sijaitsevia resursseja välimuistina. Toisena esimerkkinä viivekriittisten palveluiden, kuten pelipalvelimien suorittaminen, voi olla yksi reunalaskennan hyödyntämisen muoto.

Tutkielman kirjoitushetkellä missään ei ole reunalaskentaa laajamittaisessa käytössä. Reunalaskennan käyttöönottoa rajoittaa kokonaisvaltaisten ratkaisujen puute, sisältäen järjestelmän ja reunalaskentaa hyödyntävät ohjelmistot.
Reunalaskennan eteen on tehty paljon tutkimustyötä, mutta suuri osa siitä keskittyy matalamman tason yksityiskohtiin ja mekanismeihin. 
Tällaisia ovat esimerkiksi ohjelmiston jakaminen etänä suoritettaviin osiin, sekä näiden osien siirrettävyyden kannattavuuden päättely suoritusaikana.


Tässä tutkielmassa reunalaskentaa käsitellään reuna-arkkitehtuurien avulla.
Tutkielmaan on valittu reuna-arkkitehtuurit, joiden tavoitteena on reunajärjestelmän toteuttaminen osaksi mobiiliverkkoja.
Käsittely keskittyy siis reunalaskennan tuottavan järjestelmän käsittelyyn.
Aiheen rajaamiseksi matalamman tason toiminnallisuuksia ei käsitellä. 
Reuna-arkkitehtuurit ovat luoteeltaan ehdotuksenomaisia ratkaisumalleja reunalaskennan tuottavalle järjestelmälle. 
Ehdotukset kuvaavat reunajärjestelmän ja mobiiliverkon mekanismeja, jotka mahdollistavat reunalaskennan tuottamisen mobiiliverkossa toimiville mobiililaitteille. 

Mobiiliverkkoon sijoittuvan reunajärjestelmän ymmärtämiseksi luvussa \ref{perusteet} määritellään keskeiset käsitteet, sekä täsmennetään tämän tutkielman kontekstin mukainen määritelmä reuna-arkkitehtuurille.
Keskeiset käsitteet sisältävät reunalaskennan toimintaympäristön kuvaukset alkaen asiakaslaitteesta edeten reunan määrittelyn kautta mobiiliverkkoihin ja päättyen pilveen.
Mobiiliverkkojen yhteydessä kuvataan LTE-tyyppisen mobiiliverkon toiminta yleisellä tasolla.

Reunalaskennan toteuttaminen vaatii erilaisten teknologioiden käyttöönottoa. Luvussa \ref{mahd} käsittelemme reuna-arkkitehtuuri ehdotuksien yhteydessä ilmenneet teknologiat. Ensimmäisenä käsiteltävänä teknologiana on pilvipalveluidenkin tuottamiseen käytettävät virtuaalikoneet.
Virtuaalikoneiden jälkeen käsitellään SDN, jonka tavoittena on mahdollistaa verkkoliikenteen ohjelmallisen ohjaamisen helpottaminen. 
Viimeisenä käsitellään NFV, joka pyrkii verkkotoiminnallisuuksien virtualisointiin.

 Luvussa \ref{ominaisuudet} kuvataan reunajärjestelmän toiminnan edellytyksenä olevat ominaisuudet yleisellä tasolla. 
Näiden ominaisuuksien pohjalta luvussa \ref{ratkaisut} käsitellään ehdotetut arkkitehtuurit. 
Reuna-arkkitehtuurien esittelyn jälkeen ehdotuksia vertaillaan ominaisuuksien toteutuksien erojen pohjalta.
Tutkielma päättyy luvussa \ref{yhteenveto} esitettävään yhteenvetoon.

