
\section{Johdanto}
Reunalaskenta (Multi-access edge computing, MEC) tuo laskentaresursseja lähemmäksi asiakasta.
Verkkoyhteyksistä puhuttaessa etäisyys tarkoittaa loogista etäisyyttä, jonka mittarina toimii verkkoyhteyden viive asiakkaan ja palvelun välillä \cite{satyanarayanan2017emergence}.
Asiakkaan lähelle tuotavien laskentaresurssien on tarkoitus mahdollistavat palveluiden tuottaminen pienemmällä viiveellä ja vakaammalla verkkoyhteydellä.

%Se tarjotaa laskentaresursseja ja muita palveluita asiakaslaitteille pienellä viiveellä.
%Reunalaskennan kohderyhmänä puhutaan pääasiassa laitteista joiden resurssit ovat rajalliset.
%Suuren osan tästä kohderyhmästä muodostavat mobiiliverkossa toimivat mobiililaitteet kuten älypuhelimet. Nämä laitteet ovat perinteisesti


Reunalaskennan kohderyhmä muodostuu pääasiassa mobiililaitteista. 
Tässä yhteydessä mobiililaitteella tarkoitetaan akkuvirralla toimivia ja langattomia verkkoyhteyksiä käyttäviä laitteita, kuten älypuhelimia.
Mobiililaskenta (mobile computing) on mobiililaitteilla suoritettavaa ohjelmistojen suoritusta. 
Mobiililaskenta on luonteeltaan rajoittuneempaa kuin kiinteällä laitteistolla suoritettava laskenta \cite{ha2013just}.
Mobiililaskentaa rajoittaa mobiililaitteen laskentaresurssien määrä, verkkoyhteyden laadun vaihtelu, mobiililaitteen kestävyys ja käytettävissä olevan energian määrä \cite{satya96}. 
Mobiililaitteiden valmistajat joutuvat tasapainottamaan laskentaresurssien määrää ja virrankulutusta yhdessä laitteen koon kanssa.
Lisäksi akkuvirran rajallisuus johtaa kompromisseihin mobiililaitteen suorituskyvyssä \cite{satya01pervasive}.

Perinteiset verkkopalvelut sijaitsevat keskitetyissä palvelinsaleissa – pilvessä.
Koska pilvi tarjoaa näennäisesti äärettömästi laskentaresursseja, se saattaa kuulostaa houkuttelevalta vaihtoehdolta siirtää laskentaa pilveen suoritettavaksi.
Asiakkaasta nähden kaukana sijaitseva pilvi sisältää muutamia ongelmia.
Asiakkaan ja pilven välinen yhteys sisältää merkittävän määrän viivettä \cite{satya09}, joten se ei sovellu reaaliaikavaatimuksen sisältävien palveluiden suorittamiseen.
WAN-verkossa esiintyvään viiveeseen ei juurikaan enää voida vaikuttaa, vaan ainoana vaihtoehtona on laskentaresurssien tuominen lähemmäksi asiakasta \cite{satya09}.
Lisäksi tietoliikenteen määrän kasvun seurauksena, kaistanleveys internetin läpi kulkevan tietoliikenteen osalta ei välttämättä kykene täyttämään palveluille asetettuja vaatimuksia. 
Esimerkki reaaliaikaisesti tehtävä kasvojentunnistus mobiililaitteen välittämästä videovirrasta (video stream) vaatii kaistanleveyttä sekä pientä viivettä.

Myös mobiililaitteiden suorituskykyyn liittyvien kasvavien odotuksien seurauksena, laskennan siirtäminen mobiililaitteilta muualle suoritettavaksi on nostanut reunalaskennan varteenotettavaksi ratkaisuvaihtoehdoksi.
Reunalaskenta muun muassa tarjoaa mobiililaitteille mahdollisuuden siirtää resurssi-intensiivistä laskentaa lähellä sijaitsevalle reunalaskennasta vastaavalle yksikölle. Laskennan siirrolla tarkoitetaan etälaskentaa (offload).
Etälaskennan motivaationa voi toimia esimerkiksi akkuvirran säästäminen tai suoritusajan lyhentäminen. Tämä tulee kuitenkin laskennan siirtoon kuluvan ajan ja energian kustannuksella. 
Reunalaskentaa voi hyödyntää myös muihin käyttötarkoituksiin.
Se mahdollistaa palveluiden tuottamisen lähempänä asiakasta.
Reunalaskennan tarjoamana palveluna voi olla esimerkiksi reunalla sijaitsevien resurssien hyödyntäminen välimuistina.
Toinen esimerkki reunalaskennan mahdollistamasta palvelusta voisi olla jonkin viivekriittisen palvelun, kuten pelipalvelimen, suorittaminen reunalla sijaitsevilla resursseilla.



Tutkielman kirjoitushetkellä reunalaskentaa ei ole yleisesti tarjolla.
Reunalaskennan käyttöönottoa rajoittaa kokonaisvaltaisten ratkaisujen puute.
Reunalaskennan käyttöönotto edellyttää reunalaskennan tarjoavan reunajärjestelmän kehittämistä ja rakentamista.
Reunajärjestelmä koostuu laitteisto- ja ohjelmistokomponenteista.
Reunalaskennan eteen on tehty paljon tutkimustyötä, mutta suuri osa siitä keskittyy matalamman tason yksityiskohtiin ja mekanismeihin. 
Tällaisia ovat esimerkiksi ohjelmiston jakaminen etänä suoritettaviin osiin, sekä näiden osien siirrettävyyden kannattavuuden päättely suoritusaikana. 

Tämä tutkielma käsittelee reunajärjestelmiä reunalaskenta\hyp arkkitehtuurien avulla. Tällaiset arkkitehtuurit kuvaavat reunajärjestelmän toimintaa korkealla tasolla ja antavat yleiskuvan reunajärjestelmän keskeisimmistä ominaisuuksista.
Tutkielmaan on valittu reunalaskenta\hyp arkkitehtuurit, joiden tavoitteena on reunajärjestelmän toteuttaminen osaksi mobiiliverkkoja.
%Käsittely keskittyy siis reunalaskennan tuottavan järjestelmän käsittelyyn.
Nämä arkkitehtuuriehdotukset kuvaavat reunajärjestelmän ja mobiiliverkon mekanismeja, jotka mahdollistavat reunalaskennan tuottamisen mobiiliverkossa toimiville mobiililaitteille.
Tutkielman tavoitteena on tunnistaa reunalaskenta\hyp arkkitehtuureissa esiintyvät ominaisuudet. Tunnistettujen ominaisuuksien avulla arkkitehtuuriehdotuksien erojen jäsentäminen on helpompaa. 

Tutkielma jäsentyy seuraavasti. Luvussa \ref{perusteet} esitellään reunajärjestelmään liittyvät keskeiset käsitteet. 
Keskeiset käsitteet sisältävät määritelmän reunajärjestelmän toimintaympäristölle ja reunalaskentaan liittyvää terminologiaa.
Lisäksi, koska tutkielman kohteena on nimenomaan mobiiliverkkoihin sijoittuvat reunajärjestelmät, esitellään LTE-tyyppisen mobiiliverkon rakenne ja toiminta yleisellä tasolla.
Reunalaskennan toteuttaminen vaatii erilaisten teknologioiden käyttöönottoa. Luvussa \ref{mahd} käsitellään reuna-arkkitehtuuriehdotuksien yhteydessä ilmenneet teknologiat. Ensimmäisenä käsiteltävänä teknologiana on pilvipalveluidenkin tuottamiseen käytettävät virtuaalikoneet.
Virtuaalikoneiden lisäksi käsitellään SDN, jonka tavoitteena on  verkkoliikenteen ohjelmallisen ohjaamisen helpottaminen. 
Viimeisenä käsitellään NFV, joka pyrkii verkkotoiminnallisuuksien virtualisointiin.
Luvussa \ref{ominaisuudet} kuvataan reunalaskenta\hyp arkkitehtuureista identifioidut ominaisuudet.
Luvussa \ref{rakenne} kuvataan reunajärjestelmien rakennetta sekä arkkitehtuurin vaikutus rakenteeseen.
Ehdotetut reunalaskenta\hyp arkkitehtuurit esitellään luvussa \ref{ratkaisut}. Arkkitehtuurien esittelyn jälkeen kuvataan ehdotuksien eroavaisuudet tunnistettujen ominaisuuksien pohjalta. Lisäksi käsitellään tutkielmassa käsiteltyjen arkkitehtuurien yhteensopivuus ETSI (European Telecommunications Standards Institute) MEC spesifikaation kanssa.
Tutkielma päättyy luvussa \ref{yhteenveto} esitettävään yhteenvetoon.

