\subsection{Hallinta}

Reunalaskenta-arkkitehtuurissa hallinnalla tarkoitetaan sellaisia toimia, jotka mahdollistavat, ylläpitävät ja säätelevät asiakaslaitteen ja reunasolmun välistä kommunikaatiota. Hallinnosta vastaavana entiteetin tai entiteettien keskeinen tehtävä on erilaisiin tapahtumiin reagoiminen. Yleisimmin arkkitehtuurien yhteydessä käsiteltävät tapahtumat ovat reunalaskennan aloittaminen, laskennan migraatio ja kommunikaatioväylän avaaminen. Tämän lisäksi hallinnollisiin toimiin oletetaan reunalaskennan resurssien varaaminen sekä niiden optimointi. Näitä ei kuitenkaan korkealla arkkitehtuuritasolla kuvata.


