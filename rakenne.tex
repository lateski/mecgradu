\subsection{Rakenne}
Reunan rakenne on osittain arkkitehtuuristen komponenttien määräämä hierarkia. On kuitenkin huomattava että osa reuna-alustoista on niin vapaamuotoisia että varsinaisia rajoitteita ei ole, mutta periaatteen vuoksi tässä tutkielmassa noudatetaan arkkitehtuurin esittäjän ehdottomaa rakennetta, mikäli sellainen löytyy.

fog,
2-tier
3-tier

Reunan rakenne on yksi keskeisimpiä ominaisuuksia joita reunajärjestelmällä on. Reunan rakenteella tarkoitetaan tässä yhteydessä reuna-arkkitehtuurin asettamia rajoitteita reunasolmujen ja reuna-alustan hallinnollisten komponenttien sijaintien osalta. 
Arkkitehtuurin näkökulmasta sijainnit ovat tyypiltään suhteellisia, eli kuvataan komponenttien suhteita toisiinsa.
Tämän lisäksi on olemassa konkreettiset sijainnit, joihin arkkitehtuuri ei juurikaan ota kantaa.
Moni ehdotetuista reunaratkaisuista kuitenkin implisiittisesti esittävää jonkinlaisen konkreettisen rakenteen järjestelmälle. 

Rakenteen ominaisuuksiin kuuluu järjestelmän hajautuneisuus. Hajautuneisuuden ja reunasolmulla sijaitsevien resurssien määrää voisi ajatella verrannollisena. 

Cloudletit eivät ota kantaa sijaintiinsa \cite{satyanarayanan2017emergence}

%Mikä on reunan rakenne
%implisiittinen
-kolmikerros
-kaksikerros
-alue
-
%arkkitehtuurin
