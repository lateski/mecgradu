%Yleiset periaatteet ja vaatimukset
ETSI MEC spesifikaation teknisiä vaatimuksia esittelevä dokumentti \cite{etsitechreq} määrittelee reunalaskennalle yleiset periaatteet ja konkreettisemmin määriteltyjä toiminnallisia vaatimuksia. 
Yleiset periaatteet ovat korkean tason tavoitteita reunajärjestelmälle.
Reuna-arkkitehtuurien kannalta merkityksellisimmät periaatteet ovat NFV yhteensopivuus, liikkuvuuden tukeminen ja käyttöönoton riippumattomuus.
Toiminnallisuudet vaatimukset on jaettu useampaan kategoriaan, jotka esiteltiin kappaleessa \ref{etsi}.
Ainoastaan osa toiminnallisista vaatimuksista on relevantteja arkkitehtuureja tarkasteltaessa ja suurin osa näistä vaatimuksista koskee reuna-alustan ohjelmallisia toimijoita, joihin ei reuna-arkkitehtuureissa oteta kantaa.
Lisäksi suurin osa vaatimuksista on ainoastaan täsmennyksiä yleisiin periaatteisiin.
Seuraavaksi käydään läpi yleisten periaatteiden toteutuminen ehdotetuissa reuna-arkkitehtuureissa.

Tässä tukielmassa käsiteltyjen reuna-arkkitehtuurien yhteydessä ei esiintynyt NFV yhteensopivuutta rajoittavia tekijöitä. Eli kaikki käsitellyt arkkitehtuurit olivat pääsääntöisesti NFV yhteensopivia.
CONCERT oli ainoa reuna-arkkitehtuuri joka nimenomaisesti rakentui NFV:n varaan. Muissa arkkitehtuureissa NFV:tä pidettiin enemmänkin mahdollisuutena.
Esimerkiksi SCC:n yhteydessä esitettiin, että tukiasemaan sidottuja resursseja voisi käyttää myös tukiasemien omien toimintojen tuottamiseen NFV:n avulla. 


Perinteisesti mobiiliverkoissa liikkuvuuden tukeminen tarkoittaa, että handover tapahtumat eivät ilmene asiakaskerroksella, esimerkiksi puheluissa ja tietoliikenneyhteyksissä.
Samaa palveluvaatimusta on luonnollista olettaa myös reunajärjestelmältä. 
Käytännössä reunajärjestelmän tulee vähintää ylläpitää asiakaslaitteen ja reunapalvelun välistä yhteyttä.
Tämän lisäksi palvelun laadun takaamiseksi kyseeseen tulee myös reunapalveluiden siirto.
Liikkuvuuden tukeminen reunajärjestelmissä koostuu siis yhteyden ylläpidosta ja reunapalveluiden siirtelystä.
Tässä tukielmassa SMORE, MobiScud ja FMC esittivät mekanismin yhteyden ylläpitämiseksi asiakaslaitteen liikkuessa verkossa.
SMORE ja MobiScud toteuttivat yhteyden ylläpitämisen SDN reititysmuutoksilla.
FMC puolestaan nojaa erilliseen sessiotunnisteeseen, joka häivyttää IP-osoitteiden muutokset.
Lähes kaikki reuna-arkkitehtuurit tiedostivat tarpeen reunapalveluiden siirtelyn tarpeelle.
SMORE:a lukuunottamatta kaikki arkkitehtuurit olettivat virtuaalikoneiden live migraatiota jossain muodossa.  
ETSI:n MEC referenssiarkkitehtuuri olettaa reunajärjestelmän toteuttavan Cloudlet VM handoff kaltaista toiminnallisuutta \cite{etsirefarch}.
On kuitenkin täysin reunapalveluiden tyypistä riippuvaista, kuinka liikkuvuuteen on kannattavaa reagoida \cite{etsitechreq}. 

Reunajärjestelmän yleisiin periaatteisiin kuuluu käyttöönoton riippumattomuus (deployment independence) \cite{etsitechreq}.
Käyttöönoton riippumattomuudella tarkoitetaan, että reunajärjestelmän käyttöönottajalla olisi mahdollisuus sijoitella reunajärjestelmän osat omien vaatimuksiensa mukaan \cite{etsitechreq}. Pääasiassa tämä koskee reunasolmujen sijoittelua. Tässä tutkielmassa reunasolmujen sijoittelua on käsitelty implisiittisenä ominaisuutena, jonka määrittelevinä ominaisuuksina toimivat integraation tyyppi ja kommunikaatio. 
Teoriassa eniten vapauksia reunajärjestelmän toteuttamiseen tarjoavat läpinäkyvää integraatiota toteuttavat SMORE ja MobiScud.
Nämä järjestelmät edellyttävät ainoastaan yhden yhteyspisteen mobiiliverkkoon ja ovat muuten täysin vapaasti järjesteltävissä.
Suoran integraation reuna-arkkitehtuureissa edellytetään jotain tiettyä rakennetta. 
SCC edellyttää reunaresurssien sijoittamisen tukiasemiin. SCC kuitenkin loogisesti erottaa reunajärjestelmän ja mobiiliverkon haarauttamalla reunapalveluille tarkoitetun tietoliikenteen erillisellä monitoritoiminnallisuudella.
Toinen suoraa integraatiota edustava järjestelmä, CONCERT, edellyttää hierarkista asettelua reunaresursseille. CONCERT:n NFV painotteisuus kuitenkin takaa toiminnallisuuksien vapaan sijoittelun. CONCERT tarjoaa SCC:hen verrattuna enemmän vapauksia resurssien sijoitteluun.
FMC, ainoana epäsuorana itengraationa, tarjoaa myös vapauden sijoitella reunajärjestelmän, mutta ainoastaan mobiiliverkon ulkopuolelle.
FMC:tä ei siis voi pitää erityisen riippumattomana, koska reunapalveluita tarjoavat resurssit joudutaan sijoittelemaan pakettiverkon yhdyskäytävien läheisyyteen.

%Referenssi-arkkitehtuuri
