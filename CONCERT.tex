\subsection{CONCERT}
Nykyinen LTE televerkko koostuu joukosta laitteita, joilla on hyvin spesifit tehtävät. CONCERT \cite{liu2014concert} pyrkii seuraavan sukupolven televerkkoihin keskittyvässä ratkaisussaan vähentämään tehtäväspesifien laitteiden tarvetta. 
CONCERT arkkitehtuuri koskee siis pääasiassa seuraavan sukupolven televerkkoja, vaikkakin sen ratkaisut mahdollistavat tuen myös vanhemmille teknologioille. CONCERT esittää uudenlaista toiminnallisuuksien toteuttamismallia, joka korvaisi valmistajakohtaisia telelaitteita (kuten tukiasemat) virtualisoiduilla ja ohjelmistopohjaisilla ratkaisuilla. NFV ja SDN käyttöönotto toimii siis tämänkin ehdotuksen selkärankana.
Verkon toimintojen virtualisointi mahdollistaa laitteiston siirtämisen kauemmaksi reunasta, joka osaltaan vähentää hajautettavan laitteiston määrää. CONCERTin ideana on, että hajautetuksi jäisi telelaitteistosta ainoastaan välttämättömin osa, eli radiorajapinnan mahdollistava RIE (Radio Interfacing Equipment) ja hierarkisesti jaetut reunalaskentayksiköt. Televerkon muita toiminnallisuuksia voitaisiin keskittää ja tarjota virtuaalisina ohjelmistoteutuksina. 

Conductor on CONCERT arkkitehtuurissa hallinnollisessa keskiössä ja se vastaa päällysverkon hallinnasta.
CONCERT arkkitehtuurissa control ja data planet on erotettu toisistaan ja Conductorin tehtävänä esittää data planella esiintyvät fyysiset resurssit virtuaalisina resursseina.
Reunalaskentayksiköiden resurssien jakaminen sekä niiden välisten SDN kytkimien kautta muodostettujen yhteyksien hoitamisesta vastaa conductor.
Conductorin toiminta on toistaiseksi kuvattu vain korkealla tasolla, joten sen toteutustekniset ratkaisut ovat avoimia.

Vaikka laitteiston virtualisointi ja palveluiden keskittäminen kustannuksien säästämiseksi kuulostaakin houkuttelevalta, on kesittämisessä myös omat ongelmansa.
Etenkin liiallinen keskittäminen siirtää verkon toiminnallisuuksia kauemmaksi reunasta, joka osaltaa johtaa korkeampiin latensseihin ja heikentää palvelun laatua. Vaikka conductor onkin esitetty yksittäisenä loogisena entiteettinä, sen toiminnallisuuksia on mahdollista porrastaa ja hajauttaa paremman skaalautuvuuden ja pienempien latenssien tavoittamiseksi.

CONCERTin ratkaisu reunapalveluiden tuottamiseksi on hierarkinen kolmeen tasoon jaettu arkkitehtuuri.
\textit{Paikalliset} reunalaskentaresurssit sijaitsevat kaikkein lähimpänä verkon reunaa, esimerkiksi RIE:n kanssa samassa sijainnissa sijaitseva palvelin. Paikallisen reunalaskentayksikön laskentaresurssit ovat rajalliset ja sen tehtävänä onkin suorittaa ainoastaan kaikkein tiukimman aikavaatimuksen laskentaa.
\textit{Alueelliset} reunalaskentaresurssit kattavat jonkin pienen alueen reunalaskenta tarpeita ja korkeimman tason \textit{keskus} reunalaskentaresurssit kattavat jonkin suuremman alueen resurssi intensiivistä reunalaskentaa.
CONCERT:ssa reunalaskentaresurssien jakamisesta vastaa conductorin sisäinen komponentti LCM (Location-aware computing management).


 Reunalaskennan osalta CONCERTissa resurssit on jaettu hierarkisesti kolmeen tasoon, siten että lähimpänä käyttäjää on \textit{paikalliset} palvelimet, jotka on tarkoitettu kaikkein tiukimman aikavaatimuksen sovelluksille. Seuraavat kaksi tasoa, \textit{alueellinen} ja \textit{keskus (central)}, kasvattavat palvelinresurssien määrää, mutta ovat kauempana reunasta. Reunalaskentayksiköt voivat olla yhteydessä toisiinsa, jolloin ne mahdollistavat nopeat M2M (Machine-to-Mahcine) yhteydet. Tämä mahdollistaisi muunmuassa autojen välisen kommunikaation reunaverkon välityksellä. 

Toistaiseksi CONCERT on korkean tason suunnitelma ja se käyttää hyväkseen vielä olemassa olemattomia teknologisia ratkaisuja kuten tukiasemien virtualisointia. Arkkitehtuuri mahdollistaa erilaisien ratkaisujen toteuttamisen, eikä aseta tiukkoja rajoitteita resurssien tai hallinnon sijoittelun suhteen.

