% --- Template for thesis / report with tktltiki2 class ---
% 
% last updated 2013/02/15 for tkltiki2 v1.02

\documentclass[finnish]{tktltiki2}

% tktltiki2 automatically loads babel, so you can simply
% give the language parameter (e.g. finnish, swedish, english, british) as
% a parameter for the class: \documentclass[finnish]{tktltiki2}.
% The information on title and abstract is generated automatically depending on
% the language, see below if you need to change any of these manually.
% 
% Class options:
% - grading                 -- Print labels for grading information on the front page.
% - disablelastpagecounter  -- Disables the automatic generation of page number information
%                              in the abstract. See also \numberofpagesinformation{} command below.
%
% The class also respects the following options of article class:
%   10pt, 11pt, 12pt, final, draft, oneside, twoside,
%   openright, openany, onecolumn, twocolumn, leqno, fleqn
%
% The default font size is 11pt. The paper size used is A4, other sizes are not supported.
%
% rubber: module pdftex

% --- General packages ---

\usepackage[utf8]{inputenc}
\usepackage[T1]{fontenc}
\usepackage{lmodern}
\usepackage{microtype}
\usepackage{amsfonts,amsmath,amssymb,amsthm,booktabs,color,enumitem,graphicx}
\usepackage[pdftex,hidelinks]{hyperref}

% Automatically set the PDF metadata fields
\makeatletter
\AtBeginDocument{\hypersetup{pdftitle = {\@title}, pdfauthor = {\@author}}}
\makeatother

% --- Language-related settings ---
%
% these should be modified according to your language

% babelbib for non-english bibliography using bibtex
\usepackage[fixlanguage]{babelbib}
\selectbiblanguage{finnish}

% add bibliography to the table of contents
\usepackage[nottoc]{tocbibind}
% tocbibind renames the bibliography, use the following to change it back
\settocbibname{Lähteet}

% --- Theorem environment definitions ---

\newtheorem{lau}{Lause}
\newtheorem{lem}[lau]{Lemma}
\newtheorem{kor}[lau]{Korollaari}

\theoremstyle{definition}
\newtheorem{maar}[lau]{Määritelmä}
\newtheorem{ong}{Ongelma}
\newtheorem{alg}[lau]{Algoritmi}
\newtheorem{esim}[lau]{Esimerkki}

\theoremstyle{remark}
\newtheorem*{huom}{Huomautus}


% --- tktltiki2 options ---
%
% The following commands define the information used to generate title and
% abstract pages. The following entries should be always specified:

\title{Reunalaskenta arkkitehtuurit}
\author{Lauri Vene}
\date{\today}
\level{Pro gradu -tutkielma}
\abstract{Tiivistelmä.}

% The following can be used to specify keywords and classification of the paper:

\keywords{reuna, pilvi, tietojenkäsittelytiede}

% classification according to ACM Computing Classification System (http://www.acm.org/about/class/)
% This is probably mostly relevant for computer scientists
% uncomment the following; contents of \classification will be printed under the abstract with a title
% "ACM Computing Classification System (CCS):"
% \classification{}

% If the automatic page number counting is not working as desired in your case,
% uncomment the following to manually set the number of pages displayed in the abstract page:
%
% \numberofpagesinformation{16 sivua + 10 sivua liitteissä}
%
% If you are not a computer scientist, you will want to uncomment the following by hand and specify
% your department, faculty and subject by hand:
%
% \faculty{Matemaattis-luonnontieteellinen}
% \department{Tietojenkäsittelytieteen laitos}
% \subject{Tietojenkäsittelytiede}
%
% If you are not from the University of Helsinki, then you will most likely want to set these also:
%
% \university{Helsingin Yliopisto}
% \universitylong{HELSINGIN YLIOPISTO --- HELSINGFORS UNIVERSITET --- UNIVERSITY OF HELSINKI} % displayed on the top of the abstract page
% \city{Helsinki}
%


\begin{document}

% --- Front matter ---

\frontmatter      % roman page numbering for front matter

\maketitle        % title page
\makeabstract     % abstract page

\tableofcontents  % table of contents

% --- Main matter ---

\mainmatter       % clear page, start arabic page numbering


\section{Huomioita}
\begin{itemize}
\item reuna
\item reunasolmu
\end{itemize}

Tutkielmassa puhutaan mobiililaitteesta, mutta monet toiminnallisuudet ovat sovellettavissa kaikenlaisiin mobiileihin laitteisiin, esimerkiksi kannettaviin tai älykkäisiin ajoneuvoihin. 

Listan Mobile Computingin neljästä pääasiallisesta rajoitteesta.\\
\textit{Unpredictable variation in network quality, lowered trust and robustness of
mobile elements, limitations on local resources imposed by
weight and size constraints, and concern for battery power
consumption} [13]

[13]M. Satyanarayanan, "Fundamental Challenges in Mobile Computing," Proc.
15th ACM Symp. Principles of Dist. Comp., Philadelphia, PA, May, 1996



\section{Johdanto}


\section{Reunalaskennan perusteet}

\subsection{Motivaatio}
Reunalaskennan ideana on täydentää ja avustaa resurssiköyhiä asiakaslaitteita. Mobiililaitteisiin voitaisiin teoriassa lisätä enemmän resursseja, mutta ne tulisivat kannettavuuden ja käyttöajan kustannuksella. 
Avustamiseksi voidaan ajatella esimerkiksi tilanne, jossa akkuvirtaa säästääkseen, mobiililaite välittää resurssi intensiivisen laskutoimituksen reunasolmulle laskettavaksi. Reunasolmun voi ajatella kevennetyksi versioksi palvelimesta joten sen rajoitteet ovat hyvin erilaiset kuin asiakaslaitteen.
Täydentävä toiminta reunalaskennan avulla tarkoittaa asiakaslaitteen resurssin puutetta suoriutua jostakin tehtävästä. Esimerkiksi muistin riittämättömyys kuvankäsittelyyn. Tällöin voitaisiin esimerkiksi ottaa etäkäyttöyhteys reunasolmuun jolla käsittely tehdään. Asiakaslaitteelle jäisi tässä tilanteessa tehtäväksi ainoastaan esittää reunasolmun tilaa käyttäjälle.


Satyanarayana \cite{RefWorks:doc:5a65a2cee4b0cb152cfb50e7} esitti artikkelissaan pervasive computing esimerkkejä jokapaikan tietotekniikasta. Ympäristöön sijoitettujen laitteiden yhteistoiminnan avulla, asiakkaalle voidaan tarjota parempaa ja täsmällisempää palvelua. Yhtenä palvelun laadun ehtona on kyky ennakoida asiakkaan toimintaa.
Nykyään mobiililaitteilla on mahdollista hyödyntää langattomia yhteyksiä.
Suurin osa palvelinresursseista ja palveluiden tuottamiseen käytettävästä tiedosta  on keskittyneenä pilveen. Käytännössä minkä tahansa palvelun käyttäminen mobiililaitteella edellyttää yhteyttä näihin pilvipalveluihin. Pilvipalveluiden ylläpitäminen  

Toinen painopiste on siinä että tieto seuraa käyttäjää. Esimerkiksi pöytäkoneelta mobiililaitteeseen.(Satyanarayanan, 2001).
Cyber foraging, on termi jota käytetään kuvaamaan paradigmaa jossa laite etsii ympäristöstä hyödynnettävää tietoa ja avustajia/korvikkeita. Avustajan (surrogate) rooli on täydentää lähtökohtaisesti resurssirajallista laitetta, esimerkiksi suorittamalla laskentaa, jotta asiakaslaite voisi esimerkiksi säästää akkua.
Tähän toimintaan liittyy kuitenkin useita haasteita. Esimerkiksi kuinka asiakaslaite löytää avustajan? Mitäs jos avustaja on ruuhkautunut? Kuinka avustaja alustetaan ja kauanko siinä kestää? 
Nämä ovat keskeisiä kysymyksiä myös reunalaskennassa. 
Vastuun jakaminen asiakaslaitteen sekä reunanklusterin välillä on riippuvainen siitä, kuinka paljon avustusta asiakaslaite tarvitsee. 
Toiseen ääripäähän vietynä asiakaslaite on niin sanottu kevyt asiakaspääte (thin client), jolla ei olisi resursseja juurikaan mihinkään. Tällainen asiakaslaite joutuisi jakamaan kaiken laskennan eteenpäin avustajalle. 
Tämän kaltainen asiakaslaite olisi riippuvainen reunan mahdollisuuksista suorittaa palveluiden vaatimia toiminnallisuuksia. 
Seuraavana askeleena kohti itsenäisempää suoriutumista olisi asiakaslaite, joka pystyy osittain tarjoamaan käyttäjälle palveuita. Tämä laite tarvisi reunaklusterilta avustusta ainoastaan joissain tapauksissa. 
Viimeisenä toimijana olisi kokonaan itsenäinen asiakaslaite, joka tarvitsisi reunalta ainoastaan palveluita täydentäviä ominaisuuksia.
Tämä laite saattaisia turvautua reunalaskentaan esimerkiksi jos akku on vähissä, tai laitteella itsellään ei ole kaikea tarvittavaa tietoa laskennan suorittamiseksi.

\subsection{}
\begin{itemize}
\item Mistä reunalaskenta koostuu? (Suurimmat toimijat, keskeisimmät toiminnot)
\item Mikä on MEC?
\item Reunalaskenta vai reunapalvelu?
\item MCC vai MEC?
\item Pilvi vai palvelinkeskus
\item Mitä reunalaskenta on?
\item Miksei siirretä laskentaa pilveen?
\item Mitkä ovat mobiilin ongelmat nykyisellään?
\item Mitkä ovat reunalaskennan haasteet?
\end{itemize}


Pilvipalvleulla tarkoitetaan palvelua, joka sijaitsee internetissä. Palvelut tarjotaan käyttäjälle verkkoyhteyden välityksellä.
Pilvipalvelut sisältävät usein suuria määriä laskenta- ja tallennusresursseja. Palvelut ovat myös usein runkoverkossa kiinni, jolloin niiden voidaan ajatella sijaitsevan internetin "keskustassa".
Yleensä palveluita ylläpidetään keskitettyinä korkeintaan muutamaan eri konesaliin.
Pilvipalvelun palvelun kohde on usein asiakaslaite (UE, user equipment), joka sijaitsee internet-topologian näkökulmasta lehtisolmussa. Etäisyys 

Reunalaskenta on yksi hajautetun laskennan muoto jossa 

Mobiililaitteiden yleisiä ominaisuuksia ovat resurssien vähyys ja akkuvirran rajallisuus. Mobiililaitteiden käyttö on myös usein riippuvaista langattomista verkkoyhteyksistä.
Palveluiden toiminta mobiililaitteilla on siis riippuvainen näiden kolmen ominaisuuden asettamista rajoista.
Laskentaresurssien lisääminen johtaisin lyhyempään käyttöaikaan akkuvirralla. Suurempi akku mahdollistaa pidemmän käyttöajan, mutta se tekisi laitteesta suuremman. Akun kokoa ja laskentaresurssien määrää pyritäänkin tasapainottamaan.
Voidaan pyrkiä minimoimaan laitteen virrankulutus esimerkiksi laittamalla laitteeseen heikkotehoisempi suoritin. Tämä näkyy siinä, millaisia palveluita mobiililaitteella voidaan tarjota.
Esimerkiksi kuvankäsittelyä tai muuta raskaampaa laskentaa vaativaa toiminnallisuutta ei voida tällaisella laitteella tehdä. 
Seuraava vaihtoehto olisi lähettää laskentaa tehokkaammille laitteille pilveen. Laskennan siirtäminen vie aikaa ja tällöin ei voida tarjota kovin reaaliaikaisia palveluita.
Siirrettyyn laskentaan kuluva aika koostuu pääasiassa verkon viiveestä, siirrosta ja itse laskennasta. 
Kokonaisuudessa siirtämiseen ja laskentaan kuluvan ajan määrään vaikuttaa pilven sijainti, verkon ruuhkaisuus, verkon kapasiteetti, sekä käytössä olevan laskentakapasiteetin määrä.
Reunalaskenta on konsepti, jonka avulla laskenta voidaan tuoda lähemmäksi käyttäjää.

Reunalaskennassa (MEC,Mobile Edge Computing vai MCC Mobile Cloud Computing?) on tarkoitus tuoda palvelinresursseja lähemmäksi käyttäjää ? reunalle.
Tässä kontekstissa reunalla tarkoitetaan käyttäjän ja pilven väliin jäävää tilaa. TCP/IP-mallissa sovellustasolla olevia toimintoja ei siis esiinny tällä välillä.
Reunalaskenta siis mahdollistaa palveluiden tuottamisen lähempänä käyttäjää. Lähempänä on hieman harhaan johtava termi, koska mikä tahansa pilveä lähempänä oleva palvelu on lähempänä, eikä siis välttämättä konkreettisesti lähellä. 

Reunalaskennalle ei vielä ole olemassa kokonaisvaltaista arkkitehtuuria.
Ongelmakentän voi jakaa karkeasti kahteen osaan. Fyysiseen arkkitehtuuriin ja sovellustason arkkitehtuuriin. Nämä ovat toisistaan riippuvaisia.
Arkkitehtuuriratkaisut ovat riippuvaisia tarjottavista palveluista. Toiset arkkitehtuuriratkaisut tukevat toisia palveluita paremmin kuin toiset, kompromisseilta on siis vaikea välttyä.






\section{Suurimmat osatoimijat}
\subsection{eNodeB}
eNodeB on televerkossa tukiasemakontekstin kokonaisuus. Sen tehtäviin kuuluu teleradiotoiminnan järjestäminen. Mukaan lukien puhelinkeskuksien kanssa kommunikointi.

\subsection{Asiakaslaite}
UE (User Equipment) on yleisnimitys asiakaslaitteelle, joka hyödyntää pilven (cloud) ja reunan (edge) palveluita tietoliikenneyhteyksien avulla.
Usein käyttäjälaitteen esimerkkinä toimii jokin mobiililaite kuten puhelin, mutta myös esimerkiksi auto. 

\subsection{Edge cloud}
Edge cloud on yleisnimitys reunapalveluiden tarjoamiseen tarkoitettuille toimijoille.
Riippuen arkkitehtuurista reunapilvi koostuu yksittäisistä toimijoista tai reunaa lähellä olevista klustereista.
Edgen on mahdollista tarjota palveluita pienemmillä viiveillä ja suuremmilla tiedonsiirtokapasiteeteillä verrattuna perinteisiin pilvipalveluihin. 

\subsection{Cloud}

\subsection{Reunasolmu}
Tässä tutkielmassa reunasolmulla viitataan yksittäiseen reunalla sijaitsevaan
palvelinklusteriin, joka tuottaa asiakkaille reunapalveluita. Edge Cloud koostuu reunasolmujen joukosta.

Asiakaskohtaiselle reunainstanssille ei ole mitään vakiintunutta nimeä.
Cloudlet on yksi ehdotettu toteutustekniikka tällaiselle asiakaskohtaiselle reunalla sijaitsevalle virtuaali-instanssille \cite{RefWorks:doc:5a65a533e4b0451a693fbf8e}.

\section{Ominaisuudet}

Reunalaskennan keskeisin tarkoitus on laskennan siirtäminen reunalla toimivalle reunalaskentaklusterille. Reunaa voidaan lähestyä pilven ja käyttäjälaitteiden puolelta. Käyttäjälaitteiden, kuten älypuhelimien laskentateho on suhteellisen heikkoa, lisäksi ne ovat akkuvirrasta riippuvaisia. 
Mobiililaitteen käyttöajan pidentämiseksi voidaan pyrkiä tekemään mahdollisimman vähän akkuakuluttavaa laskentaa paikallisesti, siirtämällä sitä reunalaskentaklusterille (Etsi se lähde jossa verrataan tietokoneita ja mobiililaitteita - eroa oli yhden kertaluokan verran).
Esimerkiksi verkkoliikenne mobiililaitteen ja kohdepalvelimen välillä on merkittävä viive toisi reunaklusteri palvelun huomattavasti lähemmäksi ja pienentäisi viivettä palvelussa. Viiveen pienenemisen seurauksena monet reaaliaikaisuuttaa tai nopeaa reagointia vaativat palvelut ovat mahdollisia. Lisäksi verkon viiveestä tai ruuhkasta johtuen, mobiililaitteella on usein huomattavasti nopeampi yhteys fyysisesti lähelle itseään verra

Lisäksi pilven tai konesalien suunnasta asiaa lähestyttäessä runkoverkko tukkeutuu. Siirtämällä osan palveluvastuusta reunalle, runkoverkon rasitteen tulisi ainakin periaatteessa pienentyä. 


\subsection{Etälaskenta}
Etälaskennan toteuttamiseksi tarvitaan vastaukset seuraaviin kysymyksiin. Mitä siirretään ja minne siirretään?

MOCAssa oli selitetty kuinka muodostetaan yhteys in-network cloudiin.

Etälaskenta voidaan karkeasti jakaa kahteen tyyppiin: Binääriseen ja osittaiseen \cite{mao17}. \textit{Onkohan ok lainata surveytä?}. Binäärisessä ohjelmasta suoritetaan selkeitä kokonaisuuksia joko reunasolmulla tai asiakaslaitteella. Osittaisessa etälaskennassa suoritusta siirretään dynaamisesti reunasolmulle. 

Offloading on varmaan samankaltainen termi.
Reunalla suoritettava etälaskenta saattaa siirtää ohjelman suorituksen kokonaan tai osittain asiakaslaitteelta reunasolmulle. Laskennan tulos lähetetään takaisin asiakaslaitteelle ja sitä käytetään osana muuta laskentaa. 
Laskennan hallinta suoritetaan asiakaslaitteella. Etälaskentaa motivoi raskaiden operaatioiden siirtäminen asiakaslaitteelta reunalle. Erityisesti mobiililaitteilla akkuvirran säästäminen on keskinen tekijä. Etälaskennan kannattavuus puhtaasti akkuvirran näkökulmasta muuttuu kannattavaksi, kun suoritettavan ohjelman lähettäminen ja tuloksen vastaanottaminen kuluttavat vähemmän akkua kuin ohjelman suorittaminen paikallisesti asiakaslaitteella.
Todellisuudessa pelkästään akkuvirran säästäminen ei riitä, sillä muuten laskentaa voitaisiin siirtää pilveen. QoS kuitenkin heikkenee, mikäli laskennan suorittamiseen kuluva aika pitenee huomattavasti.
Reunalle on teoriassa nopeampi yhteys ja nopeampi vastaus. Voitaisiin siis laskea että suoritusajassa mitaten ohjelman siirtäminen reunalle on kannattavaa kun suoritettavan ohjelman lähettäminen palvelimelle, sen suorittaminen ja tuloksen vastaanottaminen kestävät vähemmän aikaa kuin ohjelman suorittaminen paikallisesti. 
Ongelmana on että suorituksien aikavaatimus ei ole eksakti vaan ainoastaan arivoitavissa. Lisäksi suoritusaikainen aika-arvion tekeminen vie myös aikaa. 
Ajan ja akkuvirran säästämiseksi tehtävät toimet ovat siis keskeisimmät haasteet etälaskentaa toteutettaessa. Näiden käsittelyä ei tämän enempää tässä tutkielmassa käsitellä niiden monimutkaisuuden vuoksi.


\subsection{Etäinstanssi}
Etäinstanssissa asiakaslaitteella on yhteys reunalle. Asiakaslaitteelle tulee ainoastaan näkymä palvelun tilasta omalle laitteelleen. Vastaavasti kuin ottaisi SSH tai VNC yhteyden toiselle laitteelle. 

-Tarkista oliko jossain järkevää lähdettä tähän
\subsection{Migraatio}
Reunalaskennan migraatiolla tarkoitetaan asiakaslaitteeseen liittyvän tilan tai laskennan siirtämistä reunasolmulta toiselle.
Handoff/-over on mobiiliverkoissa yleisesti ilmenevä tilanne jossa, mobiililaitteen yhteys siirtyy tukiasemalta toiselle.
Reunalaskennassa handover tehdään reunasolmulta toiselle. Reunalaskennan
toteutustavasta riippuen, saatetaan tarvita niin sanottu live migraatio. Live migraatiossa suorituksen alla oleva sovellus tai virtuaalikone siirretään suoritusalustalta toiselle.
Tavallisesti live migraatiota käytetään palvelinkeskusympäristössä virtuuaalikoneiden siirtämiseen suorituksen aikaiseen siirtämiseen. Live migraation tavoitteena on minimoida virtuaalikoneen käyttöön kohdistuva käyttökatkon kesto.
Live migraatio toimii siten, että siirrettävää virtuaalikonetta aletaan kopioimaan kohdelaitteelle.
Koska kyseessä on suoritusaikainen kopiointi, tilan kopiointi sisältää tallennustilan ja muistin kopioinnin.
Mikäli palvelinkeskuksessa on jaettu levypalvelin, riittää ainoastaan muistin kopiointi. Tämä huomiona lähinnä siksi, että voidaan olettaa että reunasolmuilla ei ole jaettua levypalvelinta, jolloin joudutaan kopioimaan myös tallennustila. 
Kun virtuaalikoneen tila on saatu kopioitua uudelle alustalle, joudutaan kopioimaan kopioinnin aikana virtuaalikoneen tilaan tapahtuneet muutokset.
Kopiointia jatketaan iteroiden, kunnes päästään tilaan, jossa muutoksien määrä kopio-iteraatiota kohden ei enää pienene.
Tällöin alkuperäinen virtuaalikone pysäytetään ja viimeisten muutoksien kopioinnin aikana virtuaalikone ei ole käytettävissä. Tämän jälkeen migratoitu virtuaalikoneinstanssi on käytettävissä uudella alustalla. \cite{ha2015adaptive}

Reunalaskennan ja perinteisen palvelinympäristön vaatimukset live migraatiota kohtaan ovat hieman erilaiset.
Päällimmäisenä erona on migraatioon käytettävän kaistan suuruus. Palvelinsaleissa yhteysnopeudet ovat suuria ja etäisyydet verrattain lyhyitä. Lisäksi palvelinsaliympäristössä migraatioita on mahdollista tehdä koordinoidusti ilman aikarajoitteita.
Reunasolmujen välillä olevien yhteyksien nopeudet saattavat vaihdella suuresti.
Pitkä kopiointiaika johtaa pidempään käyttökatkokseen ja täten palvelun laadun heikkenemiseen. Kopiointiajan minimoimiseksi on pyrittävä pitämään siirrettävän datan määrä mahdollisimman pienenä. Reunasolmujen migraatiotarpeeseen vaikuttaa suuresti käyttäjien liikkuminen verkossa. Voidaan kuvitella tilanne jossa aamulla kaupungin keskustaan saapuvat työmatkailijat aiheuttavat "migraatiotulvan", joka ruuhkauttaa reunasolmujen käytössä olevat kaistat.  

Virtuaalikone-aihioihin perustuvassa järjestelmässä ainoastaan virtuaalikoneeseen tehdyt muutokset siirretään \cite{RefWorks:doc:5a65a533e4b0451a693fbf8e}. Näin säästetään migraation aikana siirrettävän datan määrä.
Mikäli on tiedossa, minne käyttäjä on siirtymässä, voitaisiin migraatio tehdä suoraan kohteena olevalle reunasolmulle.
Palvelun laadun takaamiseksi migraation ennakointi on tärkeää. Mitä aikaisemmin aie siirtyä toiselle reunasolmulle tiedetään sen paremmin siihen keretään valmistautumaan. 
Migraation kesto on riippuvainen siirrettävän datan määrästä, sekä
Mikäli käyttäjä haluaa keskeyttää reunapalvelun käytön saatetaan tila tai laskenta siirtää käyttäjän laitteelle tai jättää reunalle odottamaan.

\cite{ha2015adaptive} Tukivat että migraatio virtuaalikone-aihioilla + muutoksien siirroilla aiheuttaa noin 1s downtimen hitaahkolla verkolla. Tavoitteena oli minimoida siirrettävän datan määrä ja tutkia handoff+ migraation aiheuttaman käyttökatkoksen pituutta. Testi tehtiin cloudleteillä.

MobiScudissa ja SMOREssa esitetään että käytetään livemigraatiota, mutta ei sen tarkemmin pureuduta ongelmaan. Todetaan vain että käytetään live migraatiota
\subsection{Integraatio mobiiliverkkoihin}
SMOREssa ratkaisu on että SMORE nuuskii/monitoroi LTE/EPC kommunikointia ja injektoituu väliin.

\subsection{IP-verkko}
Service discovery IP-verkossa.
\subsection{Kommunikaatio asiakaslaitteen (UE) ja Reunasolmun kanssa}
\subsection{Hallinta}
Sisältää palveluiden etsimisen (Service Discovery) ja palveluihin ohjauksen esim Software Defined Networkingin avulla (SDN).

\section{Reunan rakenne}
Rakenteeseen vaikuttaa ehkä eniten seuraavat kaksi valintaa
\begin{itemize}
\item Kerroksittain
\item Lähelle vähän vai kauemmaksi enemmän
\end{list}
Molempiin ratkaisuihin liittyy omat haasteensa. Etenkin voidaan olettaa että lähelle hajalleen aiheuttaisi huomattavasti enemmän ylläpitotyötä.

\section{Esitetyt ratkaisut}
Reunalaskennasta löytyy valtavasti artikkeleita, joissa esitetään reunalaskentaan liittyvien haasteiden ratkaisuja. Ratkaisut ovat yksittäisi tapauksia eivätkä sinällään arkkitehtuuriratkaisuja. 
\subsection{Cloudlet}
\subsection{FMC – Follow me cloud}
\subsection{Small Cell Cloud (SCeNB)}
\subsection{MobiScud ja SMORE}
\subsection{Concert}
\subsection{ETSI reference architecture}

\subsection{Yhteenveto}

 
\bibliographystyle{apalike}

\bibliography{lahteet}

\lastpage

% --- Appendices ---

% uncomment the following

% \newpage
% \appendix
% 
% \section{Esimerkkiliite}

\end{document}
