\subsection{Small Cell Cloud}
Small Cell Cloud (SCC, pienisoluinen pilvi?) perusidea on reunapalveluiden integroiminen osaksi LTE-verkkoinfrastruktuuria. 
Reunaopalveluita tuotettaisiin tukiasemiin (Base Station) sijoitetuilla palvelinresursseilla.
LTE mahdollistaa heterogeenisten verkkojen luomisen. Matkapuhelinverkkojen tapauksessa tällä tarkoitetaan sitä, että yksittäisen solun sisällä saattaa olla pienempiä soluja.
Verkko saattaa siis koostua useammasta kerroksesta erikokoisia soluja. 
Kun solut ovat muuttumassa pienemmiksi tukiasemat tulevat fyysisesti lähemmäksi käyttäjää.

SCC tarkoituksena on hyödyntää solujen pienentymistä, tuomalla niihin reunapalveluiden tuottamisen mahdollistavaa teknologiaa.
SCC reunalaskentapalvelut tuotetaan SCeNBce klustereissa. 
SCeNBce (Small-cell eNodeB computing-enhanced) on palvelinresursseilla varustettu eNodeB tukiasema \cite{lobillo15scc}.
Verrattuna cloudletteihin, SCeNBce:ssä palvelinresurssit ovat tiiviisti osana tukiasemaa.

SCM (Small Cell Manager) on SCC arkkitehtuurissa hallinnasta vastaava komponentti \cite{gambetti15dist}. Normaalisti LTE verkossa keskeinen hallinnosta vastaava komponentti on Mobility Management Entity (MME). Sen tehtäviin kuuluu on hoitaa muunmuassa MME:ltä toiselle tapahtuvat handoverit ja asiakkaiden autentikointi. Täten on luonnollista että SCM liitetään MME:n yhteyteen, jossei suorasti niin ainakin jonkin yhteyden välityksellä. Eräs ehdotettu tapa toteuttaa SCM onkin integroida se suoraan MME:n yhteyteen jolloin tämä komponentti vastaa sekä reunapalveluista ja LTE-verkon hallinnasta\cite{lobillo15scc}. SCM tehtävänä on organisoida resurssien utilisointia SCeNBce klustetereiden välillä. Tähän kuuluu esimerkiksi palvelinresurssien sammuttaminen mikäli ne eivät ole käytössä ja mahdollisesti yliutilisoidulta SCeNBce:ltä kuorman siirtäminen toiselle SCeNBce:lle. Tämän lisäksi esimerkiksi virtuaalikoneiden migraatiopäätöksien tekeminen on SCM tehtävä.

Koska SCC:n tapauksessa palveluiden tuottamiseen käytettäisiin matkapuhelinverkkoa, rajaisi se myös asiakaslaitteet älypuhelimiin ja muihin tätä verkkoa käyttäviin laitteisiin.
Reunapalveluiden sitominen tiukasti matkapuhelinverkkoon vaikuttaa erikoiselta.
Ratkaisulla on hyvät ja huonot puolensa. Reunapalveluiden tarjoamisen muihin verkkoihin, esimerkiksi WiFi yhteydellä ei onnistu suoraan ehdotetulla 

SCC:n arkkitehtuurin haasteena saattaa olla niiden ylläpidettävyys, jossa matkapuhelinverkko-operaattori on vastuussa reunasolmun ylläpidosta, koska se on tiukasti sidoksissa tukiasemaan\cite{lobillo15scc}.
Tukiasemaan tiukasti sidotulla ratkaisulla on kuitenkin mahdollista hyödyntää laajemmin tukiaseman ominaisuuksia, kuten radioyhteyksiä toisiin tukiasemiin.
SCeNBce tukiasemien hankkiminen vaatii operaattoreilta investointeja sekä lisää ylläpitotyön määrää. (Näiden vuoksi ehkä kantsis miettii, että miksi se on niin tiukasti kiinni siinä televerkossa. Muutenkin useampi operaattori samalla alueella tarkoittaa että jokaisella on myös omat reunapalvelut, joka taas meinaa kokonaiskuvassa että sama palvelu pitää duplikoida monelle operaattorille.)

\subsubsection{Kommunikointi reunapalveluun} \label{GTP}
SCC pohjaisessa järjestelmässä tavallinen verkkoon menevä tietoliikenne ja reunapalvelulle menevä tietoliikenne on ehdotettu eroteltavaksi toisistaan \cite{puente15seamless}.
Eristyksen seurauksena LTE-A arkkitehtuuriin ei tarvitse koskea muutoin kuin lisäyksien osalta. 
Liikenne SCeNBce:llä tarjottuihin reunapalveluihin on tehtävä tarkoitusta varten olevan rajapinnan kautta.
Asiakaslaitteelta tuleva liikenne reititetään SCeNBce:llä. 
Pakettien reitittämiseksi SCeNBce joutuu päättelemään, onko paketin määränpää reunapalvelu vai tavallinen tietoliikenne.

 Koska LTE-tukiasemat eivät toimi IP tasolla, ei reunapalvelulta tuleva pakettiliikenteen sisältämä asiakaslaitteen IP-osoite riitä identifioimaan kohdelaitetta. 
Yleisesti LTE verkossa, internetistä asiakaslaitteelle suuntautuvan liikenteen osalta pakettien ohjaamiseksi, käytetään GTP-tunnelia. GTP-tunnelilla tietoliikenne saadaan ohjattua oikealle tukiasemalle.
Tukiaseman ja asiakaslaitteen välisen kommunkaatioväylän selvittämiseksi joudutaan tekemään TEID (Tunnel endpoint IDentifier) ja DRB-ID (Data Radio Bearer ID) muunnos.

Vastaava ongelma on myös reunapalvelulta asiakaslaitteelle suuntautuvassa tietoliikenteessä.
SCC:n yhteydessä ehdotetussa ratkaisussa reunapalvelulta tulevan liikenteen yhdistäminen asiakaslaitteen IP-osoitteeseen voidaan tehdä käyttäen TEID perustuvaa ohjaustaulua.
TEID on asiakaskohtaisen GTP-tunnelin indetifioiva tunniste.
TEID:stä ja asiakaslaitekohtaisesta IP-osoitteesta voidaan muodostaa SCeNBce:lle reititystaulu. 
Reititystaulun avulla voidaan ohjata IP-osoitte muuntaa DRB liikkenneväyläksi ja ohjata tietoliikenne asiakaslaitteeseen.

