\subsection{Integraatio mobiiliverkkoihin}

Useat reunalaskentaan liittyvät arkkitehtuurit on suunniteltu integroitaviksi osaksi mobiiliverkkoa.
Motivaatio on ilmeinen, koska lähes kaikki mobiiliverkossa toimivat laitteet kuuluvat reunalaskennan kohderyhmään.
Reunalaskentatoiminnallisuutta lisätessä olemassa olevaan järjestelmään, yhdeksi suurimmista tekijöistä tulee hinta. 
Kuten aiemmin mainittu, mobiiliverkon laitteisto on usein suljettu, eikä siihen välttämättä ole mahdollista tehdä muutoksia tai laajennoksia.
Tämä tarkoittaa että olemassa olevaa laitteistoa jouduttaisiin korvaamaan uudellaa, mikäli reunalaskentatoiminnallisuutta halutaan lisätä. Jokaisella erilaisella ratkaisumetodilla on kuitenkin sekä hyvät että huonot puolensa.

Tavat joilla reunalaskenta-arkkitehtuurit integroituvat osaksi mobiiliverkkoa voidaan jakaa kolmeen pääryhmään

\begin{itemize}
\item \textbf{Suorat integraatiot} sisältävät uusien toimintojen lisäämistä osaksi olemassa olevaa arkkitehtuuria. Tämänkaltainen ratkaisu edellyttää muutoksia myös olemassa olevien komponenttien toimintaan. 
\item \textbf{Epäsuorat itegraatiot} eivät edellytä toiminnallisia muutoksia mobiiliverkon toimintoihin.
\item \textbf{Läpinäkyvät ratkaisut} vaativat muutoksia mobiiliverkkoon, mutta eivät vaadi muutoksia olemassa olevien komponenttien toimintaan.
\end{itemize}

Suorat integraatiot olettavat lähes koko koko verkon uudistamista. Nämä ratkaisut tähtäävät pääasiassa viidennen generaation mobiiliverkkoihin, joiden spesifikaatio on vielä vaiheessa, joten nämä ratkaisut rakentuvat osittain oletuksien päälle. 

Epäsuoraan itegraatioon kuuluvat ratkaisut, jotka eivät ole suorassa yhteydessä mobiiliverkkoon, vaan ratkaisevat ongelmat ulkoisten toimijoiden avulla. Mobiiliverkon ulkopuoliset ratkaisut käyttävät ratkaisut sijoittavat reunasolmut välittömästi P-GW:n läheisyyteen, jolloin asiakaslaitteen ja  Esimerkiksi FMC olettaa mobiiliverkon pysyvän pääpiirteittäin samanlaisena, mutta hajautuvan ainakin P-GW:den osalta horisontaalisesti, siten että mobiiliverkosta pääsee useampaa reittiä ulkoverkkoon.

Läpinäkyvät ratkaisut sisältävät lisäyksiä olemassa olevaan verkkorakenteeseen, esim lisäämällä 