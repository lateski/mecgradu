
\subsection{Integraatio mobiiliverkkoihin}
\begin{quote}
Tarkista onko käsitelty myös niin että laitteistovaatimukset tulevat esiin. Esimerkiksi vaatiiko reunalaskennan toteuttaminen erikoisvalmisteisia laitteita vai selvitäänkö tavallisilla palvelimilla?
\end{quote}


Useat reunalaskentaan liittyvät arkkitehtuurit on suunniteltu integroitaviksi osaksi mobiiliverkkoa.
Motivaatio on ilmeinen, koska lähes kaikki mobiiliverkossa toimivat laitteet kuuluvat reunalaskennan kohderyhmään.
Reunalaskentatoiminnallisuutta lisättäessä olemassa olevaan järjestelmään, yhdeksi suurimmista tekijöistä tulee hinta. 
Kuten aiemmin mainittu, mobiiliverkon laitteisto on usein suljettu, eikä siihen välttämättä ole mahdollista tehdä muutoksia tai laajennoksia.
Tämä tarkoittaa että ainakin osa olemassa olevasta laitteistosta jouduttaisiin korvaamaan uudella, mikäli reunalaskentatoiminnallisuus halutaan lisätä.
Tämän vuoksi on esitetty tekniikoita joiden avulla laitteiston uusimiselta voidaan välttyä, tai ainakin minimoida korvattavan laitteiston määrä.
Yksi erottavata tekijä on reunalaskentaratkaisun sidonnaisuus mobiiliverkkoon. Tämä tarkoittaa että löyhästi sidonnaisten ratkaisujen tuottajana voi jokin kolmas osapuoli, mutta yleisesti tämä myös tarkoittaa että reunapalvelu tuotetaan kauempana reunasta.
Jokaisessa ratkaisussa on siis hyvät ja huonot puolensa. 

Tavat joilla reunalaskenta voidaan lisätä osaksi mobiiliverkkoa voidaan jakaa kolmeen pääryhmään

\begin{itemize}
\item \textbf{Suorat integraatiot} sisältävät uusien toimintojen lisäämistä osaksi olemassa olevaa arkkitehtuuria. Tämänkaltainen ratkaisu edellyttää muutoksia myös olemassa olevien komponenttien toimintaan. 
\item \textbf{Epäsuorat itegraatiot} eivät edellytä toiminnallisia muutoksia mobiiliverkon toimintoihin.
\item \textbf{Läpinäkyvät integraatiot} vaativat muutoksia mobiiliverkkoon, mutta eivät vaadi muutoksia olemassa olevien komponenttien toimintaan.
\end{itemize}

Suorat integraatiot edustavat niiden ratkaisujen joukkoa jotka muokkaavat olemassa olevan mobiiliverkkoarkkitehtuuria.
Näiden ratkaisujen voidaan olettaa olevan kalleimpia, koska yhteistoiminnallisten toimijoiden lisääminen olemassa olevaan infrastruktuuriin vaatii muidenkin toimijoiden päivittämistä.
Pääasiassa nämä ratkaisut tähtäävät lisäämään reunalaskentaa viidennen generaation mobiiliverkkoihin, mutta myös LTE:hen pohjautuvia ratkaisuja on esitetty.
Koska viidennen generaation määrittelytyö on vielä kesken, ehdotetut ratkaisut rakentuvat osittain oletuksien päälle.
Esimerkkinä suorasta integraatiosta on CONCERT (esitellään kappaleessa \ref{concert}). Suora integraatio mahdollistaa reunalaskennan tuomisen niin lähelle reunaa kuin mahdollista.

Epäsuoralla integraatiolla tarkoitetaan ratkaisuja, joiden pääasiallinen toiminnallisuus on sijoitetu mobiiliverkko-arkkitehtuurin ulkopuolelle. 
Tällaiset ratkaisut mahdollistavat reunalaskentainfrastruktuurin tuottamisen kolmansilla osapuolilla. Esimerkkinä tällaisesta ratkaisusta on FMC (käsitellään tarkemmin kappaleessa \ref{fmc}). Haasteena tämänkaltaisissa ratkaisuissa on optimaalisen reunasolmun valinta, koska se joudutaan tekemään asiakaslaitteen antamien tietojen pohjalta. Onkin siis huomattava, että asiakaslaite joutuu osallistumaan reunalaskentaan liittyviin hallinnollisiin toimiin.

Läpinäkyvissä ratkaisuissa reunalaskennan mahdollistavat toiminnot on toteutettu siten että suoria yhteyksia mobiiliverkon toimintoihin ei ole. 
Periaate on hieman samankaltainen kuin läpinäkyvässä välipalvelimessa.
Käytännössä tämä tarkoittaa jonkinlaiseen tietoliikennemonitoriin pohjautuvaa ratkaisua. 
Monitorin tehtävänä on tarkkailla mobiiliverkon tapahtumia, sekä mahdollistaa halutun tietoliikenteen ohjaaminen reunalaskennalle.
Näiden toimintojen toteuttamisen helpottamiseksi käytössä on tai käyttöön oletetaan SDN, NFV tai molemmat.
Joskaan läpinäkyvät ratkaisut eivät ole osa mobiiliverkkoa, ne sijaitsevat sen välittömässä yhteydessä. Tämä sitoo ne osaksi mobiiliverkon infrastruktuuria ja käytännössä tämä tarkoittaa että reunapalvelualustan tuottajana on mobiiliverkon-operaattori. 
Esimerkkinä tällaisesta ratkaisusta on MobiScud, joka käydään tarkemmin läpi kappaleessa \ref{mobiscud}. Tämänkaltaiset ratkaisut ovat asiakaslaitteen näkökulmasta "näkymättömiä".
