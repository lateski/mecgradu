\subsection{Live migraatio}
%Alkukappale
%% Mitä migraatio on ja miten se liittyy handoveriin
Live migraatio, eli suorituksen aikanen siirto, tarkoittaa ohjelman tai virtuaalikoneen siirtämistä laitteistolta toiselle, siten että ohjelman tai virtuaalikoneen käyttö ei keskeydy. 
Useinmiten tällaista toiminnallisuutta käytetään palvelinkeskuksissa virtuaalikoneiden
siirtelyyn. Palvelinkeskuksissa siirtämiseen käytetään nopeita sisäisiä yhteyksiä,
jolloin tiedon siirtoon käytettävät väylät ovat nopeita. Syitä live migraation
suorittamiseksi on muutamia. Perinteisessä palvelinympäristössä live migraation
tavoitteena on virrankulutuksen optimointi, kuorman tasaaminen tai fyysisen
laitteiston huoltoon ottaminen \cite{soni2013comparative}. 

Reunalaskentaan sovellettuna live migraatiolla tavoitellaan pääasiassa laskennan 
siirtämistä asiakaslaitetta lähempänä sijaitsevalle reunasolmulle.
Reunalaskentaa suoritettaessa voi myös tulla tilanne, jossa laskentaa suorittavalla reunasolmulla ei ole resursseja laskennan suorittamiseksi. Tällöin asiakkaan virtuaalikone joudutaan siirtämään toiselle reunalaskentaa suorittavalle solmulle, jolla on enemmän resursseja.  

Live migraation ja mobiiliverkossa suoritettavan handoverin ideat ovat samankaltaisia.
Handoverilla tarkoitetaan asiakaslaitteen yhteyden siirtämistä tukiasemalta toiselle tukiasemalle. 
Handover tehdään yleensä tilanteessa jossa asiakaslaite on liikkunut mobiiliverkossa siten että se on lähempänä toista tukiasemaa. Handover voidaan myös tehdä mikäli tukiasema on ruuhkautunut. \cite{lähde}
4G LTE:ssä yhteys on tyypiltään tunnelimainen ja handoverissa tunnelin toinen pää siirretään toiselle tukiasemalle.
Yhteyden siirron alkaessa tukiasema välittää kohteena olevalle tukiasemalle asiakaslaitetta koskevat tilatiedot. Kun tilatiedot ovat välitetty, asiakaslaitteen ja tukiaseman välinen yhteys katkaistaan ja asiakaslaite muodostaa yhteyden uudelle tukiasemalle. Asiakaslaitteen tietoliikennettä puskuroidaan S-GW:n toimesta sillä välin kun yhteys on katkaistuna. Yleensä handover on kuitenkin niin nopea toimenpide että laitteen käyttäjä ei sitä huomaa.
Yhteydensiirto on suhteellisen nopea toimenpide koska siirrettävän tiedon määrä on vähäinen ja se koostuu pääasiassa erilaisista asiakaslaitteen ja tunneleiden tunnisteista.

%Miten migraatio toimii
Live migraatiossa virtuaalikone siirretään palvelimelta toiselle palvelimelle, ilman että virtuaalikoneen käyttö keskeytyy. Perinteisessä live migraatiossa siirtäminen koostuu virtuaalikoneen suoritustilan, eli prosessorin tilan, sekä muistin siirtäminen toiselle palvelimelle. Tämä toimii siten että lähtöpisteenä toimivalta laitteelta siirretään muisti ja suorittimen tila. 
Siirtäminen suoritetaan iteraatioittan siten, että ensimmäisellä iteraatiolla kaikki sivut siirretään kohdelaitteelle.
Seuraavilla kierroksilla lähtölaitteelta siirretään vain ne muistisivut joille on tapahtunut muutoksia edellisen siirron alkamisen jälkeen. 
Tätä jatketaan kunnes muutoksia sisältävien muistisivujen määrä ei vähene iteraatioittan.
Tässä vaiheessa lähtöpisteenä oleva virtuaalikone pysäytetään ja loput virtuaalikoneen muistisivut ja tilatiedot siirretään kohdelaitteelle. Tämän jälkeen virtuaalikone käynnistetään uudessa sijainnissa. 
Live migraatioon liittyy myös erilaisia optimointeja ja lähestymistapoja joita ei tässä tutkielmassa sen tarkemmin avata. 
Konesali ympäristössä on myös yleistä että varsinaista tallennustilaa ei ole tarpeen siirtää, koska se on toteutettu levypalvelimen avulla, jolloin ainoastaan yhteys täytyy siirtää. Mikäli näin ei ole, myös tallennustila tulee siirtää laitteelta toiselle. 


%Migraation uudethaasteet reunalla vs palvelinsali
\subsubsection*{Reunalaskennassa}
Reunalaskentaympäristössä suoritettavan live migraation toiminnallisuus on pääpiirteittäin sama kuin edellä kuvattu. Keskeisenä erona palvelinsaliympäristöön on juurikin suoritusympäristö. Reunalaskennassa maantieteellisesti hajautettujen reunasolmujen välillä ei välttämättä ole käytössä yhtä nopeita yhteyksiä kuin palvelinsaleissa. Tämän lisäksi reunasolmuilla ei ole nykyisten ratkaisuehdotusten mukaan käyttössään yhteistä levypalvelinta, eli myös tallennustila joudutaan siirtämään live migraation yhteydessä. 

%millainen reuna infra on? yhteysnopeudet solmujen välillä? siirrettävän datan määrä?

Reunalaskenta-arkkitehtuurien yhteydessä ei usein määritellä itse laskennan suorittamiseen käytettävää toiminnallisuutta. On kuitenkin oletettavaa että jonkinlaista virtuaalikoneisiin perustuvaa erikoistapausta käytetään sen toteuttamiseen.
Reunalaskennassa yksi keskeisimpiä virtuaalikoneisiin pohjautuvia ideoita on peräisin Cloudleteistä. 
%Siinä reunaslaskenta on toteutettu yhteisen pohjana päälle jolloin pyritään minimoimaan 


Reunalaskenta-arkkitehtuureissa mainitaan monessa otteessa reunalaskentainstanssit siirtämisen tarve, mutta vain muutamassa se on konkreettisesti toteutettu \cite{lahteet}. 


%Ehdotettujen migraatioratkaisujen nopeudet
