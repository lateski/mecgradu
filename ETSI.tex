\subsection{ETSI MEC} 
ETSI (European Telecommunications Standards Institute) on eurooppalainen telealan standardointijärjestö.
ETSI on aloittanut reunalaskennan arkkitehtuurin, sekä sen touteuttamiseksi vaadittavien toimintojen standardoimisen.

ETSIn MEC spesifikaatio määrittelee reunapalvelun tuottamiseksi vaadittavat ominaisuudet, jotka reunainfrastruktuurin tulee toteuttaa.
Spesifikaatio listaa myös mahdollisia, mutta ei vaadittuja toimintoja. 
Listaamalla vaaditut toiminnot stanrardi pyrkii yhtenäistämään reunalaskennan konsepteja.  

\subsubsection{Vaatimukset} \label{etsi}
Vaatimukset on jaettu kategorioihin sen mukaan mihin toiminnallisuuteen vaatimus liittyy.
Vaatimuksien kategoriat ovat yleiset vaatimukset (generic requirements), palvelu vaatimukset (service requirements), hallinta vaatimukset (operation and management requirements) ja viimeisenä kategoriana on kokoelma vaatimuksista, joiden teemoina ovat turvallisuus, sääntely ja veloitus (Security, regulation, charging requirements)\cite{etsitechreq}. 

Yleiset vaatimukset ovat luonteelta korkean tason kuvauksia reuna-infrastruktuurin toiminnallisuuksista. Yleiset vaatimukset kategorisoitu seuraaviin luokkiin: viitemallista, reunapalvelun sovelluksien elinkaaresta (application lifecycle),
reunapalvelun sovellusympäristöstä (application enviroment) sekä liikkuvuuden tuesta (support of mobility).
Viitemallin tulee vaatimuksien mukaan hyödyntää mukaan NFV ratkaisuja hallinnollisten toimien toteuttamiseen, mikäli mahdollista. 
ETSI:n vaatimuksen mukaan reunapalvelun tulee voida sijaita käytännössä missä tahansa kohdassa radiomaston runkoverkon reunan välillä. Sijainnin vapaus myös tarkoittaa että reuna-alusta(?) ei voi olla riippuvainen alla olevasta infrastruktuurista. 
Reunapalvelun sovelluksien elinkaareen liittyvät vaatimukset koskevat pääasiassa reunapalvelun toimijoiden oikeuksia päättää reunan sovelluksien käynnistämisestä ja sulkemista.
Reunapalveluiden sovellusympäristöä koskevissa vaatimuksissa esitetään, että sovelluksien autenttisuus ja eheys pitää pystyä varmentamaan. Sovellusympäristön täytyy myös mahdollistaa reunasovelluksen käyttöönotto toisella reuna-isännällä, ilman erikoisempaa mukautusta (without specific adaptation) \cite{etsitechreq}.
Mobiiliverkkojen ollessa kyseessä, asiakaslaitteiden liikkuminen tukiasemalta toiselle, on keskeinen käyttötapaus. Tämä heijastuu myös vaatimukseen liikkuvuuden tukemisesta reunapalveluissa.
Vaatimuksena on että asiakaslaitteen ja reunapalvelun välisen yhteyden tulee säilyä, vaikka asiakaslaite siirtyisi solusta toiseen tai asiakaslaite siirtyisi sellaiseen soluun joka on toisen reuna-isännän vastuualuetta.

Palveluvaatimukset on joukko vaatimuksia, jotka keskittyvät takaamaan reuna-infrastruktuurin perimmäiset palveluperiaatteet.
Lista palveluvaatimuksista on pitkä, joten tähän tutkielmaan on poimittu ainoastaan osa vaatimuksista. Täydellinen lista löytyy \cite{etsitechreq} julkaisusta.
Palveluvaatimukset kuvaavat toiminnallisuuksia, joiden avulla reunapalveluita voidaan tuottaa. Tällaisesta esimerkkinä tietoliikenteen reitittämiseen liittyvät vaatimukset, joiden keskeinen tehtävä on kuvata mahdolliset tietoliikennereitit reunapalveluun ja ulos reunapalvelusta. Yksi ehkä keskisimmistä palveluvaatimuksista on reuna-alusta mahdollisuus suodattaa ja muokata verkkoliikennettä. 
Lisäksi kuvataan että reunapalveluiden toimintaa ei haluta rajoittaa pelkästään asiakas-palvelu tyyppisen toimintamalliin. Reunapalveluiden tuottamiselle onkin annettu mikropalvelu -tyyppinen (microservice) kuvaus, jossa palveluntuottajat voivat toimia myös toisten reunapalveluiden kuluttajana (consumer).

\subsubsection{Viitekehys ja referenssiarkkitehtuuri}

ETSIn esittämän viitekehys esittää reunalaskentaan liittyvät korkean tason entiteetit. Nämä entiteetit on jaettu kolmeen tasoon: järjestelmä, isäntä ja verkko(system, host ja network). 
Verkkokerros sisältää verkkoyhteyksistä vastaavat entiteetit. MEC:n tapauksessa verkkokerros koostuu ainakin kolmesta osasta: sisäverkko, ulkoverkko ja televerkko.
Isäntäkerros koostuu reunapalveluiden virtualisointiin ja reunapalvelun hallinointiin keskittyvistä entiteeteistä.
Järjestelmäkerros koostuu korkeamman tason hallinnosta vastaavista elementeistä.

Referenssiarkkitehtuurissa on esitetty funktionaaliset entiteetit. Funktionaalisten entiteettien toiminta on kuvattu 
Referenssiarkkitehtuurissa reuna-isännällä (edge-host) tarkoitetaan entiteettiä, joka tarjoaa virtualisointi-infrastruktuurin, sekä reunalaskennan toteuttamiseen vaadittavat resurssit (laskenta, tallennus ja verkko).
Reuna-alustalla (Mobile edge platform) tarkoiteen sitä entiteettiä joka mahdollistaa reunapalveluiden käyttämisen. Reuna-alusta siis mahdollistaa reunapalveluiden saavuttamisen, eli käytännössä tarjoaa rajapinnan asiakaslaitteen suuntaan. Tähän kuuluu siis palvelurekisterin ylläpitäminen, reitityssääntöjen ylläpitäminen, sekä liikenteen välittäminen reunapalveluille. Reuna-alusta voi myös itse tarjota  palveluita.
Esimerkkinä tällaisesta voisi olla tunnistautumispalvelu. 
Reuna-applikaatioilla tarkoitetaan reuna-alustalla suoritettavia virtuaalikoneita, jotka suorittavat reunapalveluiden tuottamiseksi tarkoitettuja ohjelmistoja.
