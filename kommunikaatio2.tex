\subsection{Kommunikaatio}
Reunalaskennan toimiminen vaatii asiakaslaitteen ja reunajärjestelmän välille kommunikaatioväylän.
Kommunikaatioväylän tehtävänä on välittää asiakaslaitteen ja reuna-alustan välistä tietoliikennettä molempiin suuntiin.
Reunan ja asiakaslaitteen välinen kommunikaation toteutuminen vaatii suunnittelupäätöksiä, joihin reuna-arkkitehtuurit ottavat kantaa. 
Johtuen edellisessä kappaleessa käsitellystä reunajärjestelmän integroinnista mobiiliverkkoon, tietoliikenne reunajärjestelmän ja asiakaslaitteen välillä joudutaan ohjaamaan hieman perinteisistä tavoista eroavilla metodeilla. 

Tavallisesti mobiiliverkko ohjaa kaiken asiakkaslaitteelta lähtevän tietoliikenteen mobiiliverkosta ulos internettiin P-GW:n kautta. 
Mobiiliverkko ei käsittele asiakkaan tietoliikennettä IP-tasolla, vaan asiakkaan tietoliikenne EPC:n ja tukiasemien välillä tunneloidaan GTP:n (GPRS Tunnelling Protocol) avulla \cite{lobillo15scc}.
Toisin sanoen mobiiliverkko on asiakaslaitteen tietoliikenteen näkökulmasta näkymätön.
Koska reuna-arkkitehtuurien ehdotus on integroida reunajärjestelmän toimijat mobiiliverkon sisään, joudutaan mobiiliverkon sisäisiä tietoliikenteen ohjausmetodeja muokkaamaan.
Ensimmäinen ratkaistava ongelmana on siis, että kuinka asiakaslaitteen tietoliikenne voidaan ohajata mobiiliverkon sisällä sijaitsevalle reunajärjestelmälle.
Toinen ratkaistava ongelma on yhteyden säilyttäminen tilanteessa, jossa asiakaslaitteen ja mahdollisesti myös reunasolmun sijainti vaihtuvat.
%lisää metateksti loppukappaleen sisällöstä

\subsubsection*{Tietoliikenteen haarauttaminen}
%Perustoiminnallisuus eli pakettin tarkkailu "sisääntuloväylällä" jolloin voidaan ohajata tietoliikennettä
Ensimmäiseen ongelmaan on ehdotettu erilaisia ratkaisuja, joiden toimintamalli rippuu pääasiassa siitä, kuinka tiukasti reunajärjestelmä on integroitu osaksi mobiiliverkkoa.
Ongelman ratkaisussa on otettava huomioon, että tietoliikennettä on aina kahteen suuntaan. Tämän lisäksi asiakaslaitteen tietoliikenne pitää jatkossa haarauttaa suuntautumaan joko reunajärjestelmälle tai internettiin. 
Koska perinteisesti asiakaslaitteen tietoliikenne on voitu ohjata P-GW:n kautta ulos verkkoon, ei mobiiliverkon sisällä ole valmiiksi mekanismia jonka avulla tietoliikennettä voisi ohjata.
Yleinen tähän ongelmaan ehdotettu ratkaisumalli on perustunut ajatukseen tietoliikennettä tarkkailevasta entiteetistä.
Tällaisella entiteetillä olisi mahdollisuus tarkkailla asiakaslaitteiden tietoliikenteen kohdetta, esimerkiksi kohde IP-osoitetta, ja ohjata paketit tarvittaessa reunajärjestelmälle.
Kaiken muun tietoliikenteen se antaisi kulkea normaalia reittiä pitkin.

LIPA (Local IP Access) ja SIPTO (Selected IP Traffic Offload) ovat ehdotuksia mobiiliverkkoon tehtävästä reititysratkaisusta \cite{samdanis2012traffic,3gpplipa}.
LIPA ja SIPTO lisäisivät tukiasemien yhteyteen L-GW:n (Local Gateway), jonka mahdollistaisi asiakaslaitteen tietoliikentee ohjaamisen tukiaseman yhteydestä esimerkiksi reunasolmulle. tietoliikenne internettiin, ilman että tietoliikenteen tarvitsee kulkea EPC:n kautta. 
L-GW:n tarkoitus olisi siis mahdollistaa nopeammat yhteydet tukiaseman läheisyydessä sijaitsevaan verkkoon. Tätä olisi mahdollista hyödyntää reunalaskennassa, mutta se monimutkaistaisi olemassa olevan mobiiliverkon toiminnallisuutta \cite{cho2014smore}.
LIPA:a ja SIPTO:a ei varsinaisesti ole ehdotettu minkään reuna-arkkitehtuurin yhteydessä ratkaisuksi edellä kuvattuun ongelmaan. 

Reuna-arkkitehtuureissa tietoliikenteen tarkkailua mahdollistava toiminnallisuus on ehdotettu toteutettavan SDN, NFV tai jollain ratkaisu spesifillä toiminnallisuudella. 
SDN:n käyttö ratkaisuna tarjoaa tavan reitittää liikennettä mobiiliverkon sisällä. Se lisäksi mahdollistaa reitityksien muokkaamisen jolloin tiettyjen yhteyksien reittiä voidaan dynaamisesti muokata. SMORE:n yhteydessä esitetyssä ratkaisussa  E-UTRAN ja EPC:n välille sijoitetaan SDN kerros, jossa eNodeB ja EPC:n välistä tietoliikennettä voidaan monitoroida. 
Koska tällä välillä oleva tietoliikenne on GTP tunneloitua, paketteja joudutaan purkamaan ja uudelleen paketoimaan. 
Tilanteessa jossa paketin suunta on asiakaslaitteelta reunalle, paketin tunnelointi joudutaan purkamaan ja varsinainen paketti välitetään reunapalvelulle. 
Reunalta asiakaslaitteelle suuntautuvat paketit joudutaan uudelleen kapseloimaan GTP:n mukaisiksi.
Uudelleen kapselointi vaatii erinäisten mobiiliverkon sisäisten tunnisteiden seurantaa ja hyödyntämistä.
Tarkempi kuvaus toiminnasta annetaan SMORE:n käsittelyn yhteydessä kappalessa \ref{smore}.
Ratkaisu spesifien toiminnallisuuksien yhteydessä noudatetaan hyvin samankaltaista toimintamallia. Esimerkiksi Small Cell Cloud (käsitellään kappaleessa \ref{scc}) ratkaisun yhteydessä pakettien monitorointi on sijoitettu tukiasemiin, josta GTP tunnelointi alkaa. Täten paketit voidaan monitoroida ja välittää tarpeen mukaan reunasolmuille, ennen GTP tunnelointia.

Monitorointia suorittavalla entiteetillä on myös muuta käyttöä kuin asiakasliikenteen ohjaaminen reunalle. Kontrollitason pakettien tarkkailu paljastaa esimerkiksi tukiasemien välisen handoverin alkamisen, jota voidaan hyödyntää laskennan live migraation käynnistämiseen.

\subsubsection*{Yhteyden säilyvyys}


Toiseen ongelmaan, eli liikkuvuuden varmistamiseen sekä asiakaslaitteen siirtyessä että laskennan siirtyessä.

%tällaisella entiteetillä voi olla myös muuta käyttöä, kuten tukiasema handoverin tarkkailu, jolla voidaan laukaista reunalaskennan migraatio.

%LIPTO SIPO 



